\documentclass[11pt,a4paper]{article}
\usepackage{amsmath}
\usepackage{amsthm}
\usepackage{graphicx}
\usepackage{fullpage}
\usepackage[linesnumbered, ruled, vlined]{algorithm2e}
\usepackage[noend]{algpseudocode}
\usepackage{enumerate}
\usepackage{xcolor}
\usepackage{framed}
\usepackage{bm}
\usepackage{pifont}
\usepackage{multicol}
\usepackage{lipsum}
\usepackage{tikz}
\usetikzlibrary{calc}
\usepackage{array}
\usepackage{relsize}
\usepackage{comment}
\usepackage{url}
\usepackage{caption}
\usepackage{subcaption}
\usepackage{mathtools}
\usepackage{thmtools}
\usepackage{thm-restate}
\usepackage[normalem]{ulem}
\usepackage[hidelinks]{hyperref}
\usepackage[capitalise]{cleveref}
\usepackage{xspace}
\usepackage{amssymb}
\usepackage{glossaries}
% ======================================================================
% Editing / collaboration
% ======================================================================

% Show diff (exactly one of this and the following pair of commands must be commented)
\newcommand{\add}[1]{\textcolor{blue}{#1}}
\newcommand{\del}[1]{\textcolor{gray}{#1}}

% Preview changes (exactly one of this and the previous pair of commands must be commented)
% \newcommand{\add}[1]{#1}
% \newcommand{\del}[1]{}

\newcommand{\replace}[2]{\del{#1}\add{#2}}

% Rendered comments
\newcommand{\guy}[1]{[\textcolor{blue}{\textbf{gg:}} {\footnotesize
\textcolor{purple}{#1}]}}
\newcommand{\matej}[1]{[\textcolor{blue}{\textbf{mp:}} {\footnotesize
\textcolor{teal}{#1}]}}
\newcommand{\arp}[1]{[\textcolor{blue}{\textbf{arp:}} {\footnotesize
\textcolor{brown}{#1}]}}
\newcommand{\jorge}[1]{[\textcolor{blue}{\textbf{js:}} {\footnotesize
\textcolor{violet}{#1}]}}
\newcommand{\cmt}[1]{[\textcolor{blue}{\textbf{CMT:}} {\footnotesize
\textcolor{red}{#1}]}}
\newcommand{\TODO}[1]{[{\textbf{\red{TODO:}}} {\footnotesize
\textcolor{gray}{#1}]}}

\newcommand{\ignore}[1]{}

% ======================================================================
% Formatting macros
% ======================================================================

% Some text colors
\newcommand{\blue}[1]{{\color{blue}{#1}}}
\newcommand{\red}[1]{{\color{red}{#1}}}
\newcommand{\out}[1]{{\red{\sout{#1}}}}

\newcommand{\subnetName}[1]{\textbf{\texttt{#1}}}
\newcommand{\actorName}[1]{\texttt{#1}}
\newcommand{\funcName}[1]{\textit{\texttt{#1}}}
\newcommand{\funcParam}[1]{\textit{#1}}
\newcommand{\accountName}[1]{\texttt{#1}}
\newcommand{\accountNameFull}[2]{\subnetName{#1}.\texttt{#2}}
\newcommand{\tx}[1]{\textit{#1}}
\newcommand{\funcNameFull}[3]{\subnetName{#1}.\actorName{#2}.\funcName{#3}}

% ======================================================================
% Macros for names to use in the text
% ======================================================================

\newcommand{\ipcFull}{Interplanetary Consensus\xspace}
\newcommand{\ipc}{IPC\xspace}
\newcommand{\sa}{\actorName{SA}\xspace}
\newcommand{\saFull}{Subnet Actor\xspace}
\newcommand{\gw}{\actorName{IGA}\xspace}
\newcommand{\gwFull}{IPC Gateway Actor\xspace}
\newcommand{\actor}{actor\xspace} % TODO: Get rid of these and use the word "actor" directly in text. The name is very unlikely to change.
\newcommand{\actors}{actors\xspace}
\newcommand{\pofFull}{Proof of Finality\xspace}
\newcommand{\pofsFull}{Proofs of Finality\xspace}
\newcommand{\pof}{\textit{PoF}\xspace}

% ======================================================================
% Legacy macros
% ======================================================================

%\newtheorem{observation}{Observation}
\newtheorem{claim}{Claim}
%\newtheorem{lemma}{Lemma}
%\newtheorem{corollary}[lemma]{Corollary}
\newtheorem{notation}{Notation}
\newtheorem{invariant}{Invariant}
\newtheorem{execution}{Execution}
\newtheorem{ensemble}{Ensemble}
\newtheorem*{ensemble*}{Ensemble}
\newtheorem{strategy}{Strategy}
\newtheorem*{strategy*}{Strategy}
%\newtheorem{definition}{Definition}
\newtheorem{assumption}{Assumption}
%\newtheorem{theorem}{Theorem}
\newtheorem*{theorem*}{Theorem}
%\newtheorem{proposition}{Proposition}
% \newreptheorem{claim}{Claim}
% \newreptheorem{definition}{Definition}
% \newreptheorem{lemma}{Lemma}

\newcommand{\alg}{$\mathcal A$}
\newcommand{\act}{\alpha}
\newcommand{\wrt}{w.r.t.\/}
\newcommand{\adv}{\sigma}
\newcommand{\SN}{\mathcal{SN}}
\newcommand{\PN}{\mathcal{PN}}
\newcommand{\parent}[1]{\texttt{parent}(#1)}
\newcommand{\user}{u}
\newcommand{\fil}{\textit{amt}\xspace}
\newcommand{\txwf}{$tx(\fil,\; \user,\; fee)$\xspace} %tx w\ fee
\newcommand{\txnf}{$tx(\fil,\; \user)$\xspace} %tx no fee
\newcommand{\chkp}{$\langle \texttt{chkp},\; fee \rangle $\xspace} %checkpoint
\newcommand{\pofs}{$pofs$\xspace}
\newcommand{\prop}{$\langle \texttt{prop},\; fee, \; dest \rangle $\xspace} 
\newcommand{\report}{$\langle \texttt{report},\; \pofs \rangle $\xspace} %report slash
\newcommand{\slashop}{$\langle \texttt{slash},\; \pofs \rangle $\xspace} %slash
\newcommand{\pUp}{\textsc{PropUp}}
\newcommand{\pDn}{\textsc{PropDn}}
\newcommand{\pHr}{\textsc{PropHere}}
\newcommand{\smr}{SMR\xspace}
\newcommand{\gov}{\textit{gov-acc}\xspace}
\newcommand{\pom}{\textit{PoM}\xspace} % Proof of Misbehavior
\newcommand{\verifyGfinal}[2]{\textit{verifyGlobalFinality}({#1},{#2})\xspace} % verify the global finality of (#1-state/tx) in the child subnet with assitance of (#2-proof).
\newcommand{\verifyPfinal}[2]{\textit{verifyParentFinality}({#1},{#2})\xspace} % verify the global finality of (#1-state/tx) in the parent subnet with assitance of (#2-proof).
\newcommand{\ssc}{\textit{ShouldSubmitCheckpoint}\xspace}
\newcommand{\prf}{\textit{PoF}\xspace}
\newcommand{\data}{\textit{data}\xspace}
\newcommand{\dest}{\textit{dest}\xspace}
\newcommand{\src}{\textit{src}\xspace}
\newcommand{\postoffice}{postbox\xspace}
\newcommand{\eoa}{\textit{EOA}\xspace}
\newcommand{\propagate}{\textit{propagate}}
\newcommand\T[1]{\noindent\textbf{#1}}
\newcommand{\impl}{IPC reference implementation\xspace} % name it differently?

\makeglossaries

\newglossaryentry{replica}{
    name={Replica},
    description={Description of a replica.}
}

\newglossaryentry{child}{
    name={Child subnet},
    plural={children},
    description={Description of the child subnet.}
}

\newglossaryentry{transaction}{
    name={Transaction},
    description={t}
}

\newglossaryentry{subnet}{
    name={Transaction},
    description={t}
}

\newglossaryentry{rootnet}{
    name={Transaction},
    description={t}
}

\newglossaryentry{parent}{
    name={Transaction},
    description={t}
}

\newglossaryentry{checkpoint}{
    name={Transaction},
    description={t}
}

\newglossaryentry{membership}{
    name={Transaction},
    description={t}
}

\newglossaryentry{state}{
    name={Transaction},
    description={t}
}

\newglossaryentry{participant}{
    name={Transaction},
    description={t}
}

\newglossaryentry{full node}{
    name={Transaction},
    description={t}
}

\newglossaryentry{actor}{
    name={Transaction},
    description={t}
}


\newacronym{ipc}{IPC}{Interplanetary Consensus}


\graphicspath{{figs/}}
\newcounter{myCounter}

\begin{document}

\title{\ipcFull (\ipc)\thanks{\url{https://ipc.space}}}
\date{}
\author{ConsensusLab\thanks{\url{https://consensuslab.world}}}
% \authornote{Both authors contributed equally to this research.}
% \orcid{1234-5678-9012}
% \author{G.K.M. Tobin}
% \authornotemark[1]
%\email{sgoren@campus.technion.ac.il}
% \orcid{https://orcid.org/0000-0003-2158-161X}
% \affiliation{%
%     \institution{Technion}
%     \country{Israel}
%     }

\maketitle
\begin{abstract}
Totally ordering transactions constitutes a
significant scalability bottleneck standing in the way of massive adoption of
blockchain systems, as all participants need to replicate and execute sequentially every single transaction. Layer 2 (L2) protocols aim at resolving this scalability
limitation by off-loading state and processing to a loosely coupled sub-system.

We present \ipcFull (\ipc), a new blockchain
architecture design that enhances the scaling capabilities of L2+
protocols. Users of \ipc can dynamically spawn new blockchain subsystems
(subnets) as children of any existing subnet.  \ipc is based on the design principles of
on-demand horizontal scaling. Child subnets leverage the
security of their parent subnets by periodically checkpointing their state in the parent's state.
\ipc provides native communication across
subnets within the \ipc framework.

We believe that \ipc addresses several issues with existing L2 approaches,
including limited throughput capacity, isolation from each
other, centralized components, or monolithic architectures.
% We believe, however, that \ipc's flexible architecture could be implemented in
% conjunction with many of these orthogonal solutions.
In the following, we introduce the overall system architecture, native functionality,
and design decisions of our reference implementation for child
subnets based on the BFT Trantor consensus protocol, with Filecoin as
rootnet.
\end{abstract}

% ----------------------------------------------------------------
% ----------------------------------------------------------------

\section{Introduction}
\label{sec:introduction}

A blockchain system is a platform for hosting replicated applications (represented by smart contracts in Ethereum [??] or actors in Filecoin [??]).
A single system can, at the same time, host many such applications,
each of which containing logic for processing inputs (also known as transactions, requests, or messages) and updating its internal state accordingly.
The blockchain system stores multiple copies of those applications' state and executes the associated logic.
In practice, applications are largely (or even completely) independent.
This means that the execution of one application's transactions rarely (or even never) requires accessing the state of another application.

Nevertheless, most of today's blockchain systems process all transactions for all hosted applications (at least logically) sequentially.
The whole system maintains a single totally ordered transaction log containing an interleaving of the transactions associated with all hosted applications.
The total transaction throughput the blockchain system can handle thus must be shared by all applications, even completely independent ones.
This may greatly impair the performance of such a system at scale (in terms of the number of applications).
Moreover, if processing a transaction incurs a cost (transaction fee) for the user submitting it, using the system tends to become more expensive when the system is saturated.

The typical application hosted by blockchain systems is asset transfer between users (wallets).
Asset transfers often involve other applications and may create system-wide dependencies between different parts of the system state.
In general, if users interacted in an arbitrary manner (or even uniformly at random), this would indeed be the case.
However, in practical systems, users tend to cluster in a way that those inside a cluster interact more frequently than users from different clusters.
While this ``locality" makes it unnecessary to totally order transactions confined to different clusters (in practice, the vast majority of them),
many current blockchain systems spend valuable resources on doing so anyway.

An additional issue of such systems is the lack of flexibility in catering for the different hosted applications.
Different applications may prefer vastly different trade-offs (in terms of latency, throughput, security, durability, etc...).
For example, a high-level money settlement application may require the highest levels of security and durability, but may more easily compromise on performance in terms of transaction latency and throughput.
On the other hand, one can imagine a distributed online chess platform (especially one supporting fast chess variants) whose state is mostly ephemeral (lasting until the end of the game) but which requires high throughput (for many concurrent games) and low latency (few people like waiting 10 minutes for the opponent's move).
While the former is an ideal use case for the Bitcoin network, the latter would probably benefit more from being deployed in a single data center.
\jorge{It's an okay example but just noting that concurrent games are independent and can be seen as different applications; I don't think it negates the specific point being made here.}

In the above example, one can also easily imagine those two applications being mostly, but not completely independent.
E.g., a chess player may be able to win some money in a chess tournament and later use it to buy some goods outside of the scope of the chess platform.
In such a case, few transactions involve both applications (e.g., paying the tournament registration fee and withdrawing the prize money).
The rest (e.g., the individual chess moves) are confined to the chess application and can thus be performed much faster and much cheaper (imagine playing chess by posting each move on Bitcoin for comparison).

\ipcFull (\ipc) is a system that enables the deployment of heterogeneous applications on heterogeneous underlying blockchain platforms, while still allowing them to interact in a secure way.
The basic idea behind \ipc is dynamically deploying separate, loosely coupled blockchain systems that we call \emph{subnets}, to host different (sets of) applications.
Each subnet runs its own consensus protocol and maintains its own ordered transaction log.

\ipc is organized in a hierarchical fashion, where each subnet, except for one that we call the \emph{rootnet}, is associated with exactly one other subnet called its \emph{parent}.
Conversely, one parent can have arbitrarily many subnets, called \emph{children}, associated with it.

This tree of subnets expresses a hierarchy of trust.
All components of a subnet and all users using it are assumed to fully trust their parent and regard it as the ultimate source of truth.
Note that, in general, trust in all components of the parent subnet is not required, but the parent system as a whole is always assumed to be correct (for some definition of correctness specific to the parent subnet) by its child.

To facilitate the interaction between different subnets, \ipc provides mechanisms for inter-subnet communication.
Since subnets are distributed transaction-processing systems
without an obvious single entity to submit transactions to one subnet on behalf of another subnet,
we introduce processes called \emph{\ipc agents} that read the replicated state of one subnet and submit transactions on its behalf to another subnet.
Participants running those \ipc agents get rewarded for such mediation.
Out of the box, \ipc provides several primitives for subnet interaction, such as
\begin{enumerate}
    \item Transfer of funds between accounts residing in different subnets.
    \item Saving checkpoints (snapshots) of a child subnet's replicated state in the replicated state of its parent.
    \item Submitting transactions to a subnet by the application logic of another subnet.
\end{enumerate}

The operating model described above is simple but powerful.
In particular, it enables
\begin{itemize}
    \item Scaling, by using multiple blockchain/SMR platforms to host a large number of applications.
    \item Optimization of blockchain platforms for applications running on top of them.
    \item Governance of a child subnet by its parent, by way of the parent serving as the source of truth for the child and, for example, maintaining the child's configuration, replica set, and other subnet-specific data.
    \item ``Inheriting" by the subnet of some of its parent's security and trustworthiness, by periodically anchoring its state in the state of the parent using checkpoints.
\end{itemize}

In the rest of this document, we describe \ipc in detail.
In \cref{sec:preliminaries} we define...
\TODO{Finish this when all sections are stable.}
\section{Example Use Case: On-Line Gaming Platform}
\label{sec:example-use-case}

To better understand how \ipc works and how it is useful, let us imagine an example application of a distributed on-line gaming platform.
Consider a platform where registered players meet and play games against each other, while the platform maintains player rankings.
Tournaments can be organized as well, where each participant pays a participation fee and the winner(s) obtain prize money (both in form of coins).
We now describe how \ipc could be used to build this hypothetical application in a fully distributed fashion.

\paragraph{Rootnet with all users' funds (\subnetName{L1}).}
The rootnet is used as a financial settlement layer.
Most users' coins are on accounts residing in the rootnet's replicated state.
A robust established blockchain system like Filecoin could be a reasonable candidate for use as the rootnet.
Its relatively higher latency and lower throughput (that is often the price for security and robustness) is not a practical issue,
as users will rarely interact directly with it.

\paragraph{Gaming platform as a subnet (\subnetName{L2}).}
The functionality of the gaming platform
(such as maintaining score boards, recommending opponents to players, or organizing tournaments)
is implemented as a distributed application on a dedicated subnet.
This subnet uses a significantly faster BFT-style consensus protocol (such as Trantor)
since the application needs to be responsive for the sake of user experience,
and deals, in general, with fewer funds than the rootnet (only as much as users dedicate to playing).
The replicas constituting this subnet are run by gaming clubs or even some (not necessarily all) individual players
(who do not necessarily trust each other, e.g. to not manipulate the score boards).
To have a replica in the \subnetName{L2} subnet, the club (or the player) needs to lock a certain amount of funds as collateral that can be slashed by the system if the replica misbehaves.
% The collateral / slashing mechanism is described in more detail in \Cref{sec:incentives-collateral-slashing,sec:slash}.

\paragraph{Individual games (\subnetName{L3}).}
For each individual game, a new child of the \subnetName{L2} subnet is created (\Cref{sec:create}) and acts as a (distributed) game server.
Since not much is usually at stake in a single game and only few players are involved,
the whole \subnetName{L3} subnet may even be implemented by a single server the players trust.
However, this decision is completely up to the players and they may choose a different implementation of the \subnetName{L3} subnet
when starting the game (by submitting the corresponding transactions to the \subnetName{L2} subnet).
When the game finishes, its result is automatically reported to the \subnetName{L2} subnet (\Cref{sec:cross-net-tx}), which updates the players' ratings accordingly,
and the \subnetName{L3} subnet is disposed of (\Cref{sec:remove}).

\paragraph{Player accounts.}
Each player has an account on the \subnetName{L2} subnet where they deposit funds (\Cref{sec:deposit}) from the rootnet by submitting a corresponding \subnetName{L1} transaction%
\footnote{We call a transaction submitted to the \subnetName{L1} subnet an ``\subnetName{L1} transaction''.}.
They use these funds to pay transaction fees on the \subnetName{L2} subnet and tournament registration fees.
A player can transfer funds back to their \subnetName{L1} account through a withdraw operation (\Cref{sec:withdraw}) by submitting an \subnetName{L2} transaction.

\paragraph{Tournaments.}
Tournaments can be organized using the platform, where each player registers by submitting a corresponding \subnetName{L2} transaction.
When the tournament finishes, the winner receives the prize money (obtained through the registration fees) on their \subnetName{L2} account.
One can also easily imagine that only part of the collected fees transforms to the prize, while the rest can remain in the platform
and be used for other purposes, such as rewarding the owners of the replicas running the subnet hosting the platform (i.e., the \subnetName{L2} subnet).
To stretch the example even further, one could imagine a tournament being implemented as an \subnetName{L3} subnet, while the tournament's individual games are its children (\subnetName{L4}).

\paragraph{}
This simple use case utilizes most of \ipc's features.
Throughout the rest of the document, we will use the on-line gaming platform as a running example when describing \ipc's functionality in more detail.
%\matej{This ``running example'' part is still to be added to the rest of the document. Coming soon, but I'd prefer to have some feedback on the general suitability of this example. Then I start integrating it in the text throughout the document.}
% \arp{Following discussion on integrating cross-net txs in this example, how about having a common ranking of a user across different games (e.g.: heckers, 3d chess, Xiangqi, etc.) that are in different subnets (rootnet children) and cross-net txs inform each other of match results to locally update user's ranking (and to authenticate in the first place)}
% \matej{In fact, I meant basically that above, in the paragraph about individual games (last sentence is already pointing to the section on cross-net txs.)}\akosh{I would find it interesting to discuss mechanisms to ensure trust in the results coming from the subnets, so no player can artificially boost their standing.}\jorge{the weakness of the example is really how far from typical blockchain use cases it is; but I think it's a good proxy for explaining things, and I like where this is going.}
\section{Preliminaries}
\label{sec:model}

The vocabulary used throughout this document is described in the Glossary \cite{glossary}.
The reader is assumed to be familiar with the terminology defined there.
\jorge{Despite this not, the vocabulary seems to be defined throughout the body. If we're defining in the body anyway, the appendix seems redundant -- this isn't a book.}

% \subsection{Computation and failure model}

% We model \ipc as a distributed (``message-passing") system consisting of \emph{processes} that communicate by exchanging \emph{messages}%
% \footnote{Network messages are not to be confused with Filecoin actor messages, to which this document refers as transactions.}
% over a network. 
% In practice, a process is a program running on a computer, having some state, and reacting to external events and messages received over a communication network.
% We describe processes as exemplified in \Cref{alg:process-definition}.

% \begin{algorithm}[H]
% \footnotesize
% \caption{Process definition.}\label{alg:process-definition}
%   \DontPrintSemicolon
%   \SetKwProg{Component}{$\blacktriangleright$ \bf}{:}{\KwRet}
%   \SetKwFor{UponKW}{upon}{do}{fintq}
%   \SetKw{Trigger}{trigger}
%   variableA = initial value\\
%   variableB = initial value\\ \jorge{this is kind of a weird notation, as you can't tell whether you're defining a new variable or just assigning a new value}
%   ...\\
%   \Component{process}{
%      \UponKW{event(params...)}{
%        \tcp{Logic to execute atomically}
%      }
%      \UponKW{event(params...)}{
%        \tcp{Logic to execute atomically}
%      }
%      ...
% }
% \end{algorithm}

% A process that performs all the steps exactly as prescribed by the protocols in which it is participating is \emph{correct}.
% A process that stops performing any steps (i.e., \emph{crashes}) or that deviates from the prescribed protocols in any way is \emph{faulty}.
% If a process is correct or may only fail by crashing, it is \emph{benign}.
% A non-benign process is \emph{malicious}.
% \matej{We can remove terms we end up not using...}

% In general, faulty processes can be malicious (Byzantine): we do not put any restrictions on their behavior, except being computationally bounded and, thus, unable to subvert standard cryptographic primitives, such as forging signatures or inverting secure hash functions.
% If the implementation of some component in our design requires additional assumptions on the behavior of faulty processes, they will be stated explicitly.
% % We do not make a general statement about the fault tolerance of \ipc as a system, as to how many faulty processes the system can sustain.
% % This depends on the final implementation of its components.

% We use the term \emph{participant} to describe an entity participating in the system that controls one or more processes.
% All processes controlled by one participant are assumed to be in the same trust domain, that is, they assume one another's correctness.
% For example, a participant in the child subnet will probably run multiple processes:
% one for participating in the child subnet (child replica),
% one for participating in the parent subnet (parent replica),
% and one that processes the information from the two replicas and submits transactions accordingly (\ipc agent).
% We precisely define the replicas and the \ipc agent (all of them being processes) in \Cref{sec:components,sec:smr}.
% The \ipc agent of a participant always assumes that the information it receives from "its own" child replica is correct.
% However, messages received from another participant's replica or \ipc agent are seen as potentially malicious.

% The synchrony assumptions may vary between different components of \ipc.
% We thus state those assumptions whenever necessary, when describing concrete implementations of \ipc components.

% \subsection{State machine replication (SMR) and \actors}
% \label{sec:smr}

% \paragraph{SMR and replicated state.}
% A \emph{state machine replication (SMR) system}%
% is a system consisting of processes called \emph{replicas}, each of which locally stores a copy of (or at least has access to) \emph{replicated state}
% that it updates over time by applying a sequence of \emph{transactions} to it. 
% \jorge{it's not objectively wrong, but not a huge fan of defining replicas now, after they were mentioned several times}
% Without specifying the details, we assume that any process can \emph{submit} a transaction to an SMR system (we call such a process an \emph{SMR client})
% and that this transaction will eventually be ordered and applied to the replicated state.
% We call an SMR system that is part of \ipc a \emph{subnet}.

% An SMR system guarantees to each correct replica that, after applying $n$ transactions to its local copy of the replicated state,
% the latter will be identical to any other correct replicas' copy of the replicated state after applying $n$ transactions.
% The replicas achieve this by executing an \emph{ordering protocol} to agree on a common sequence of transactions to apply to the replicated state.

% Note that replicas do not necessarily all hold the same replicated state at any instant of real time,
% since each replica might be processing transactions at a different time.
% In this context, there is no such thing as “the current replicated state of the SMR system”.
% There is only the current replicated state of a single replica.
% The replicated state of the system is only an abstract, logical construct
% useful for reasoning about transitions from one replicated state to another,
% happening at individual replicas by applying transactions (at different real times).
% When referring to a “current” replicated state, we mean the state resulting from the application of a certain number of transactions to the initial state.

% \paragraph{Actors.}
% The replicated state of an SMR system can be logically subdivided into multiple \emph{\actors}.
% A \actor is a portion of the replicated state with well-defined semantics.
% It defines the logic that a replica needs to execute when applying transactions and the new state that results from it.

% We model a \actor as a logical object in the replicated state that contains arbitrary variables representing its state.
% Its associated logic reacts to \emph{events} triggered by (1) the application of transactions or (2) the execution of other (or even own) actor logic. \jorge{why "even own"? why is that more surprising than foreign actor logic?}
% We describe actors as exemplified in \Cref{alg:actor-definition}. \jorge{Is it a definition or an example?}


% \begin{algorithm}[H]
% \footnotesize
% \caption{\actor definition}\label{alg:actor-definition}
%   \DontPrintSemicolon
%   \SetKwProg{Component}{$\blacktriangleright$ \bf}{:}{\KwRet}
%   \SetKwFor{UponKW}{}{}{fintq}
%   \SetKw{Trigger}{trigger}
%   variable = initial value\\
%   variable = initial value\\
%   ...\\
%   \Component{\actor name}{
%      \UponKW{Function(params...)}{
%        \tcp{Logic to execute}
%      }
%      \UponKW{Function(params...)}{
%        \tcp{Logic to execute}
%      }
%   }
% \end{algorithm}
% While \emph{process} denotes an instance of a program running on some physical machine,
% \actors are an abstraction over the replicated state of an SMR system and their logic is being executed by all its replicas.
% While a process can submit a transaction to an SMR system, a \actor cannot.


\paragraph{Naming.}
We assign each subnet a name that is unique among all the children of the same parent.
Similarly to the notation used for absolute paths in a file system, the name of a child subnet is always prefixed by the name of its parent.
For example, subnets $P/C$ and $P/D$ would both be children of subnet $P$.

\paragraph{Notation.} We refer to an account $a$ in the replicated state of subnet $S$ as $S.a$.
To denote a function of an actor in the replicated state of a subnet, we write $Subnet.Actor.Function$.
E.g., the \gw's function $CreateChild$ in subnet $P$ is denoted $P.\gw.CreateChild$.
We also use this notation for a transaction $tx$ submitted to subnet $P$ that invokes the function, e.g., $tx = P.\gw.CreateChild(P/C, params)$.

\paragraph{Interaction between subnets.}
In \ipc, the replicated state of one subnet must react to (changes in) the replicated state of another subnet.
As the replicated state of every subnet is distributed among its replicas and evolves independently of other subnets,
we must establish a mechanism for interactions between the states of subnets.
In particular, we must explicitly link the two replicated states of two subnets.
More precisely, for any interaction between two subnets ($A$ and $B$), define block heights $h_A$ and $h_B$,
such that $A$'s replicated state at height $h_A$ considers $B$'s replicated state to have evolved exactly until $h_B$.

\paragraph{\pofsFull.}To enable interaction between subnets, we define a \emph{\pofFull (\pof)} to be data that proves that an SMR system definitively reached a certain replicated state.
Regardless of the SMR system's ordering protocol's approach to finality (e.g., immediate finality for classic BFT protocols, or probabilistic finality in PoW-based systems),
a \pof convinces the proof's verifier that the replicated state the \pof refers to will not be rolled back.
For example, for a BFT-based SMR system, a quorum of signatures produced by its replicas can constitute a \pof.
We denote by \emph{\pof(tx)} the proof that an SMR system reached a state in which transaction \emph{tx} already has been applied.


\subsection{Representing value}

For each pair of subnets in a parent-child relationship, we assume that there exists a notion of \emph{value} (measured in \emph{coins}) common to both subnets.%
\footnote{One can easily generalize the design to decouple the use of value between a parent and its child, but we stick with using the same kind of value in both subnets for simplicity.}
Each end user of the SMR system is assumed to have a personal wallet and a corresponding account in some subnet.

We also assume that the submission, ordering, and applications of transactions is associated with a valuable cost.
Each SMR client submitting a transaction to a subnet is assumed to have an account in that subnet, from which this cost is deducted.
If the funds are insufficient, the SMR system ignores the transaction.
 % - IPC Functionality
 %  - Minimum required per subnet
 %    - Withdrawal/Deposits Interfaces
 %    - Other Operations? (Propagate?)
 %  - Enhancements
 %    - Checkpointing interfaces
 %    - Propagate
 %    - Reporting/Slashing interfaces
 %    - Atomic execution/swap, IBC-like bridges
 %  - Future stuff (google docs?)
 %    - Withdrawal at ancestor (skip parent(s)) (with timeout) etc.
\section{IPC functionality}
\label{sec:functionality}

IPC exposes the following functionalities:
\begin{itemize}
    \item Creating child subnets.
    \item Removing child subnets.
    \item Depositing coins from an account in a subnet to an account in its child.
    \item Withdrawing coins from an account in a subnet to an account in its parent.
    \item Checkpointing - including a checkpoint of a subnet's replicated state in the replicated state of its parent.
    \item Propagating cross-net-transactions - invoking smart contracts in a subnet through changes in the replicated state of another subnet.
\end{itemize}
In the following, we describe each functionality in detail.

\subsection{Creating a child subnet}

Any user of a subnet $P$ can create a new subnet $P/C$ by submitting a transaction $P.\gw.CreateChild(P/C, params)$.
This results in the creation of a new subnet actor $\sa_C$ in $P$ governing the subnet $P/C$.
The \emph{params} value describes all the subnet-specific parameters required to initialize the state of $P.\sa_C$,
such as the initial membership data, the consensus protocol to use, etc.

\subsection{Deposits}
\label{sec:deposit}

\del{
\arp{Consider need to pause/remedy subnet after deposit (e.g. collateral not enough with new supply). IPC agent should check in that case}\guy{Does this comment belong here or somewhere else?}\\
}

A deposit is a transfer of funds (of some amount \fil) from \replace{a user $\user$ account in the parent subnet to $\user$'s account}{an account \src in the parent subnet $P$ to an account \dest} in the child subnet $P/C$.
\replace{We assume that $\user$ is a participant running a parent replica, a child replica, and an \ipc agent%
\footnote{If $\user$ does not run these processes, then it contacts a trusted participant that does and that performs the deposit on $\user$'s behalf.}
}{We assume that the owner of \src is either running their own IPC Agent to perform the necessary operations described below, or uses another trusted IPC agent to act on their behalf}.
The deposit is performed as follows:
\begin{enumerate}
    \item The \replace{local \ipc agent}{owner of \src} submits \replace{to the parent \smr replica the corresponding (properly signed)}{a} transaction
    $\tx=\textit{P.\sa.Deposit}\left(\src, \fil, \dest \right)$.
    \item The parent subnet orders and executes the \emph{Deposit} transaction (provided $\src$ has enough funds) by transferring \fil from \replace{$\user$'s parent account}{\src} to the \sa (concretely, to $\dest$ account representation within the \sa). This effectively locks the funds within the \sa \dapp, until the \sa \dapp transfers it back to $\src$ during a withdrawal (see \Cref{sec:withdraw}).
    \item When the parent's replicated state that includes \replace{the transaction}{\emph{tx}} becomes final (for some SMR-system-specific definition of finality),
    \replace{the local parent replica notifies the local \ipc agent, potentially attaching a proof of finality of \prf to the notification.}{
    The \ipc agent constructs a $\pof(tx)$\footnote{The exact content of \prf for the transaction \tx depends on the implementations of the SMR systems. It might contain, for example, a quorum of replica signatures, a Merkle proof of inclusion, or even be empty.}
    }
    \item xThe IPC Agent submits a transaction $\tx' = \textit{P/C.Deposited}(\fil, \dest, \pof)$ to the child SMR system.
    \item Upon ordering $\tx'$, the replicated logic of the child SMR system mints \fil new coins and adds them to $\dest$.
\end{enumerate}

\replace{We show in \Cref{fig:deposit} t}{T}he events being produced and consumed by the deposit functionality and in Algorithm~\ref{alg:deposit} the pseudocode per component to implement the functionality.

% \begin{figure}[h]
%      \centering
%      \includegraphics[width=\textwidth]{deposit}
%      \caption{Events produced and consumed during a deposit.}
%      \label{fig:deposit}
% \end{figure}
 

\begin{algorithm}[H]
\footnotesize
\caption{Deposit operation}\label{alg:deposit}
  \DontPrintSemicolon
  \SetKwFunction{FMain}{Global}
  \SetKwProg{Pn}{Function}{:}{\KwRet}
  \SetKwInOut{Input}{input}
  \SetKwProg{Component}{$\blacktriangleright$ \bf}{:}{\KwRet}
  \SetKwFor{UponKW}{upon}{do}{fintq}

   \Component{\replace{IPC agent}{Owner of \src}}{
        submit $\tx=\textit{P.\sa.Deposit}\left( \src, \fil, \dest \right)$ to parent subnet\;
  }
   \Component{P.\sa.Deposit(\src, amt, \dest)}{
    move $\fil$ from \src to P.\sa.\textit{accounts}.\dest  \tcp*[r]{"lock" at parent}
  }
  \Component{IPC agent}{
    \UponKW{tx = P.\sa.Deposit final at parent}{
        create \prf that \tx is final at parent subnet \tcp*[r]{see \cref{sec:finality}}
        submit \textit{P/C.\gw.Deposited(amt, \dest, \pof)}
     }
  }
  \Component{\replace{Child \smr replica}{P/C.\gw.Deposited(amt, \dest, \pof)}}{
%    \UponKW{\texttt{Deposited}}{
        \replace{assert \gw.\verifyPfinal{\tx}{\prf}}{verify(\pof)}\;
        increase \dest account by \fil
%     }
  }
\end{algorithm}

\subsection{Withdrawals}
\label{sec:withdraw}

A withdrawal is a transfer of funds from \replace{a user $\user$ account in the child subnet to $\user$'s account in the parent subnet. We assume that $\user$ is a participant running a parent replica, a child replica and an \ipc agent.}{an account \src in the child subnet $P/C$ to an account \dest in the parent subnet $P$}.
The \replace{withdraw}{\emph{Withdraw}} is performed \replace{as follows}{analogously to the \emph{Deposit}, but starting at the child subnet $P/C$}:
\begin{enumerate}
  \item \replace{$\user$ triggers the}{The owner of \src submits a transaction \emph{tx =}} $\textit{P/C.\gw.Withdraw}(\src, \fil, \dest)$.
    \item The child \replace{SMR system}{subnet} orders and executes the \emph{Withdraw} transaction, burning $\fil$ funds in $\src$ (provided $\src$ has enough funds).
    \item When the child's replicated state that includes the transaction becomes final (for some SMR-system-specific definition of finality that has been defined in the SA), the \replace{local child replica notifies the local \ipc agent, potentially attaching a proof \prf that this state is final}{\ipc agent constructs a corresponding \pof and submits a transaction \textit{\tx' = P.\sa.Withdrawn(\fil, \dest, \pof)} to the parent subnet}.
    \item Upon ordering $\tx'$, \replace{the replicated logic of the parent SMR system updates the state of the \sa transferring the funds}{\emph{P.\sa.Withdrawn(amt, \dest, \pof)} verifies the \pof and transfers \emph{amt}} from \sa (concretely, to $\src$ account representation within the \sa) to $\dest$ within the parent subnet.
\end{enumerate}

 % \begin{figure}[h]
 %     \centering
 %     \includegraphics[width=\textwidth]{withdrawal}
 %     \caption{Events produced and consumed during a withdrawal.}
 %     \label{fig:withdrawal}
 % \end{figure}
\begin{algorithm}[H]
\footnotesize
\caption{Withdraw operation}\label{alg:withdraw}
  \DontPrintSemicolon
  \SetKwFunction{FMain}{Global}
  \SetKwProg{Pn}{Function}{:}{\KwRet}
  \SetKwInOut{Input}{input}
  \SetKwProg{Component}{$\blacktriangleright$ \bf}{:}{\KwRet}
  \SetKwFor{UponKW}{upon}{do}{fintq}
  % \Input{user~$\user$, amount~$\fil$, transaction \txnf}
   \Component{\replace{IPC agent}{owner of \src}}{
        submit $\tx=\textit{P/C.\gw.Withdraw}(\src, \fil, \dest)$\;
  }
   %
   \Component{\replace{Child \smr replica}{\textit{P/C.\gw.Withdraw(\src, \fil, \dest)}}}{
   \replace{\UponKW{$\tx = \textit{Withdraw}(\src, \fil, \dest)$}{
    deduct $\fil$ from \src \tcp{"burns" \fil in child}
   }}{
    deduct $\fil$ from \src \tcp{"burns" \fil in child}
   }
  }
  \Component{IPC agent}{
    \UponKW{\replace{notification of \texttt{Burned}(\tx) from child \smr replica}{tx = P/C.\gw.Withdraw(\src, \fil, \dest)} final at child}{
        create \replace{\prf that \tx is final at child \smr}{$\pof(tx)$} \tcp*[r]{see \cref{sec:finality} for details}
        submit \replace{$\tx'=\texttt{Burned}\left(\tx, \prf \right)$  to parent \smr replica}{\textit{P.\sa.Withdrawn(amt, \dest, \pof)}}
     }
  }
  \Component{P.\sa.Withdrawn(amt, \dest, \pof)}{
    \replace{assert \sa.\verifyGfinal{\prf}{\tx}}{verify($\pof(tx')$)}\;
    move \fil coins from P.\sa to \dest \tcp{"unlocks" \textit{amt} for \dest}
  }
   
\end{algorithm}
\label{enhancedFunc}

\subsection{Checkpointing} 
A checkpoint contains a representation of the state of the child \replace{SMR system}{subnet} to be included in the parent \replace{SMR system}{subnet's replicated state}. A checkpoint can be triggered by predefined events (e.g.,  periodically after a number of state updates, triggered by a specific user or set of users, etc.).
A checkpoint is \replace{performed}{created} as follows:
\replace{
\begin{enumerate}
\item When the predefined checkpoint trigger is met (the IPC Agent, monitoring the child subnet's state, is configured with the checkpoint trigger), the IPC agent queries the parent's \sa.
\item If the participant is a validator according to \sa's state, then the IPC agent queries the child SMR replica for the child's state to be represented in this checkpoint. 
\item The IPC agent creates a \prf that this updated state of the child SMR system is final, possibly compressing its representation of the state. 
\item The IPC agent evaluates a function \ssc(sa, parent's state, etc.) and decides whether the participant submits the checkpoint. If the function returns true, then the IPC agent submits a transaction $\tx=\texttt{Checkpoint}\left(\textit{state}, \prf \right)$ to the parent SMR replica.
 \item Upon ordering $\tx$, the replicated logic of the parent SMR system updates the state of the SA according to the checkpoint state, if necessary.
\end{enumerate}
}{
\begin{enumerate}
    \item When the predefined checkpoint trigger is met (the IPC Agent, monitoring the child subnet's state, is configured with the checkpoint trigger),
    the IPC agent retrieves the corresponding checkpoint data (\emph{chkp}) from the child subnet, along with the proof of its finality (\emph{\pof}).
    \item \TODO{Here the IPC Agent should decide (based on rights and some possible reward) whether to submit the Checkpoint transaction.}
    \item The IPC agent submits a transaction \emph{tx = P.\sa.Checkpoint(chkp, \pof)}.
    \item The \emph{P.SA.Checkpoint(chkp, \pof)} invocation, after verifying the \emph{\pof}, includes \emph{chkp} in its state.
\end{enumerate}
}

% \TODO{Pseudocode}
% See commented algorithm

% \begin{figure}[h]
%      \centering
%      \includegraphics[width=\textwidth]{checkpoint.png}
%      \caption{Events produced and consumed by the checkpointing functionality.}
%      \label{fig:chkp}
%  \end{figure}

\begin{algorithm}[H]
\footnotesize
\caption{Checkpoint operation}\label{alg:checkpoint}
  \DontPrintSemicolon
  \SetKwFunction{FMain}{Global}
  \SetKwProg{Pn}{Function}{:}{\KwRet}
  \SetKwInOut{Input}{input}
  \SetKwProg{Component}{$\blacktriangleright$ \bf}{:}{\KwRet}
  \SetKwFor{UponKW}{upon}{do}{fintq}

  \Component{IPC agent}{
    \UponKW{Checkpoint condition in child}{
      $chkp = $ obtain state snapshot from child\;
      create $\pof(chkp)$\;
      submit $P.\sa.Checkpoint(chkp, \pof(chkp))$
    }
  }

  \Component{P.\sa.Checkpoint(chkp, \pof(chkp))}{
    verify($\pof(tx')$)\;
    save $chkp$ in the state\;
    \TODO{Expand on this. check if the checkpoint is the latest one and use a variable to store the latest checkpoint}
  }
   
\end{algorithm}

% OLD PSEUDOCODE
% \begin{algorithm}[H]
% \footnotesize
% \caption{Checkpoint operation\TODO{Update to new terminology}}\label{alg:down}
%   \DontPrintSemicolon
%   \SetKwFunction{FMain}{Global}
%   \SetKwProg{Pn}{Function}{:}{\KwRet}
%   \SetKwInOut{Input}{input}
%   \SetKwProg{Component}{$\blacktriangleright$ \bf}{:}{\KwRet}
%   \SetKwFor{UponKW}{upon}{do}{fintq}
%    \Component{IPC agent}{
%         \If{trigger for checkpoint}{
%             \textit{SA\_state} $\gets$ query parent for \sa's state\;
%             \If{\textit{Self} \textbf{in} SA\_state.validators}{
%                 \textit{state} $\gets$ query child for state\;
%                 \textit{cState} $\gets$ \textit{compressState}(\text{state},\textit{SA\_state.latestCheckpoint})\;
%                 create \prf that \textit{cState} is final at child\;
%             }
%             $\tx=\sa.\texttt{Checkpoint}\left(\textit{cState}, \prf \right)$ \;
%             \If{\textit{Self.}\ssc(tx, SA\_state, ...)}{
%                 submit $tx$ to parent \smr replica\;
%             }
%         }
%   }
%   \Component{parent \smr replica}{
%     \UponKW{$\tx=\sa.\texttt{Checkpoint}\left(\textit{cState}, \prf \right)$}{
%         assert \sa.\verifyGfinal{\textit{cState}}{\prf}\;
%         $\sa.\textit{latestCheckpoint.update}(\textit{cState})$
%      }
%   }
% \end{algorithm}

% The above pseudo code is intentionally abstract, with a number of implementation decisions not specified, such as the main function for creating and verifying a \prf, events that trigger the creation of a new checkpoint, the compression procedure with respect to the previous checkpoint, and the \ssc function to decide whether the participant submits or not a checkpoint. 

% The above pseudo code is highly abstract, with the main function of creating and verifying a \prf not specified. Moreover, other important aspects that are not covered include specific compression mechanisms for the checkpoint data, triggering checkpoints efficiently, and particular incentives for checkpoints creation and submission. \arp{we refer to reference implementation... later in this document we list others...}
% The function \ssc comprises two aspects of the checkpointing functionality from the perspective of participants. First, it controls access to submit checkpoints, as not all subnets will define the same policy to follow when deciding the participants that are allowed to submit checkpoints. Second, it contains the implications of submitting a checkpoint transaction (i.e. the cost involved in being the submitter). For example, if only one transaction is required by any participant but the cost of submitting the checkpoint is incurred on the submitter, then there is a risk of no participant actually submitting the checkpoint if they are strictly rational. An example on the other end might be requiring all participants to submit a transaction for the checkpoint to be finalized at the parent, but this approach affects performance. \arp{We analyze and suggest later in this document multiple mechanisms to ensure through incentives that at least one rational participant will always submit the checkpoint. }

\subsection{Propagating cross-net transactions}

Unlike a "standard" transaction issued and submitted to a subnet by a user,
a cross-net transaction is issued by the replicated logic of another subnet.
Cross-net transactions are a means of interaction between smart contracts located on different subnets.

Since those smart contracts themselves are not processes (but mere parts of a subnet's replicated state),
they cannot directly submit transactions to other subnets.
IPC therefore provides a mechanism to propagate these transactions between subnets using a ``\emph{postBox}'' and an IPC agent.
In a nutshell, if a smart contract's logic produces a transaction for a different subnet,
this transaction is saved the local Gateway Actor in a buffer that we call the \emph{postBox}.
The IPC agent, monitoring the postBox, then submits the transaction to the appropriate subnet.

Since, in general, we only rely on IPC Agents to be able to submit transactions to parents or children of a subnet whose state they observe,
the IPC agent only propagates the transaction to the parent or child, depending on which is closer in the IPC hierarchy to the ultimate destination subnet.
After such ``one hop``, the transaction is again placed in the postBox of the parent / child, and the process repeats until the transaction reaches its destination subnet.

The implementation of the Gateway Actor's \emph{Propagate} function is sketched in \Cref{alg:po}.

\TODO{Formalize subnet names and explain how the "\emph{src}" is built.}

% \guy{Edge case: a leaf subnet does not have a \sa and, therefore, no \postoffice. We can consider removing the \postoffice functionality from the \sa and to deploy it as an independent \dapp that will appear only once per subnet. In this case, it needs permissions to call \sa.\verifyGfinal{\tx}{\prf} function.}

\begin{algorithm}[H]
\footnotesize
\caption{Cross-net transaction propagation functionality}\label{alg:po}
  \DontPrintSemicolon
  \SetKwFunction{FPropagate}{propagate}
  \SetKwProg{Pn}{Function}{:}{\KwRet}
  \SetKwInOut{Input}{input}
  \SetKwProg{Component}{$\blacktriangleright$ \bf}{:}{\KwRet}
  \SetKwProg{Empty}{\bf}{:}{\KwRet}
  \SetKwFor{UponKW}{upon}{do}{fintq}
  \Component{\gw.Propagate($\tx, \src, \dest, \pof)$)}{
    verify(\tx.\pof)\;
    \Case{\dest = this subnet}{
        apply \tx
    }
    \Case{\dest requires going up the tree}{
       $postBox \leftarrow postBox \cup (\tx, S/\src, \dest)$\;
    }
    \Case{\dest requires going down the tree}{
        $postBox \leftarrow postBox \cup (\tx, \src/S, \dest)$\;
    }
  }
  \Component{IPC agent}{
    \UponKW{new entry (\tx, \src, \dest) in parent.\gw.postBox}{
        Create \pof proving that \tx' has indeed been added to the list fo cross-net transactions in the subnet\;
        submit \tx', augmented by \emph{\pof}
    }   
}
\end{algorithm}

\subsection{Slashing}
\arp{leave for later on with incentives and reconfiguration? it is hard to (meaningfully) talk about this without involving these concepts. Here a first attempt though:}
Slashing is a penalty imposed on provably malicious validators. When validators of a child subnet misbehave, other participants can report the misbehavior for these malicious validators to get punished (e.g. by losing a previously collateralized amount). Contrary to misbehaviors at a subnet with no parents, where misbehavers sucessfully perform their attack without escrow available, misbehaviors at the child can be resolved at the parent subnet, provided the misbehavers have not left the subnet.  For this reason, a slash focuses on notifying the parent subnet as soon as posible, in the hope to stop an attemped attack. In particular, a slash on provably malicious validators of a child subnet is performed as follows:
\begin{itemize}
    \item When the IPC agent of a correct participant identifies slashable misbehavior at subnet $P/C$ from a set $\mathcal{M}$ of malicious validators of subnet $P/C$, the IPC agent constructs a \textit{Proof of Misbehavior} (\pom). \item The IPC agent then submits transaction $tx_P=P.\sa.Slash(\mathcal{M}, \pom)$ at the parent and $tx_C=P/C.\gw.Slash.(\mathcal{M}, \pom)$ at the child.
    \item The parent subnet orders and executes $tx_P$. Once the parent's replicated state that includes $tx_P$ becomes final, the IPC agent constructs a $\pof(tx_P)$ and submits a transaction $\>\>\>\>\>\>$$tx'_P=P/C.\gw.Slashed(\mathcal{M}, \pom, \pof(tx_P))$
    \item When the child’s replicated state that includes $tx_C$ becomes final (for some SMR-system-specific definition of finality that has been defined in the SA), the IPC agent constructs a corresponding $\pof(tx_C)$ and submits a transaction $tx’_C = P.\sa.Slashed(\mathcal{M}, \pom, PoF)$ to the parent subnet.
    \item The parent subnet, upon ordering and executing either $tx_P$ or $tx_C'$, penalizes the misbehavers and updates the state of $\sa$. 
    \item The child subnet, upon ordering and executing either $tx_C$ or $tx_P'$, updates its state to reflect the penalization at the parent. \arp{This behavior at the child can later be updated to abort if it took place with $tx_C$ and in fact $tx_P'$ will never happen (i.e. attackers were faster and left subnet on time)} 
\end{itemize}
\begin{algorithm}[H] 
\arp{this alg must be updated, but RSs to converge on text above first}
\footnotesize
\caption{Slash Functionality}\label{alg:down}
  \DontPrintSemicolon
  \SetKwFunction{FMain}{Global}
  \SetKwProg{Pn}{Function}{:}{\KwRet}
  \SetKwInOut{Input}{input}
  \SetKwProg{Component}{$\blacktriangleright$ \bf}{:}{\KwRet}
  \SetKwFor{UponKW}{upon}{do}{fintq}
  \Input{-}
  \Component{Child SMR}{
     \UponKW{Proofs of fraud \pofs generated}{
       Notify \report to IPC agent
     }
  }
  \Component{IPC agent}{
    \UponKW{\report notified by child SMR}{
        Submit \slashop to parent SMR
    }
    \UponKW{\arp{State updated after slashing}}{
      \arp{Check child SMR rules are still satisfied, remedy/close otherwise?}
    }
  }
  \Component{Parent SMR}{
        \UponKW{\slashop submitted by IPC agent}{
         Update SA state slashing/excluding participants
         Notify SA update to IPC agent
        }
  }
\end{algorithm}


\subsection{Removing a child subnet}

A child subnet $P/C$ can be removed from its parent $P$ through a transaction invoking $P.\gw.RemoveChild(P/C)$.
\matej{We will later define a mechanism to determine who has the right to do this and when.}

% OLD PSEUDOCODE
% \begin{algorithm}[H]
% \footnotesize
% \caption{\postoffice Functionality}\label{alg:po}
%   \DontPrintSemicolon
%   \SetKwFunction{FPropagate}{propagate}
%   \SetKwProg{Pn}{Function}{:}{\KwRet}
%   \SetKwInOut{Input}{input}
%   \SetKwProg{Component}{$\blacktriangleright$ \bf}{:}{\KwRet}
%   \SetKwProg{Empty}{\bf}{:}{\KwRet}
%   \SetKwFor{UponKW}{upon}{do}{fintq}
%   \Input{$\tx = \langle \data, \src, \dest, \prf \rangle$}
%   \Component{\gw.\postoffice}{
%      \UponKW{\postoffice.\propagate(\tx) }{
%        \Case{\dest in current subnet}{
%             \postoffice.\propagate\textit{HERE}(\tx)
%        }
%        \Case{\dest requires going up the tree}{
%             \postoffice.\propagate\textit{UP}(\tx)
%        }
%        \Case{\dest requires going down the tree}{
%             \postoffice.\propagate\textit{DOWN}(\tx)
%        }
%      }
%      \UponKW{\postoffice.\propagate\textit{UP}(\tx) }{
%        \If{\src not from this subnet}{
%             assert \sa.\verifyGfinal{\tx}{\prf}\tcp*[r]{the \sa that corresponds to the child subnet from which \tx comes}
%        }
%        \src.\textit{append(\gw's subnet id)}\tcp*[r]{the i.d. of the current subnet}
%        emit event \gw.\postoffice.UP$\langle \data, \src, \dest \rangle$\;
%        % $\tx \gets \langle \data, \src, \dest \rangle$\;
%        notify agent on \gw.\postoffice.UP$\langle \data, \src, \dest \rangle$
%      }
%      \tcp{\propagate\textit{DOWN}(\tx) is analogous to \propagate\textit{UP}(\tx)}
%      \tcp{\propagate\textit{HERE}(\tx) is trivial}
%   }
%   % \Component{parent \smr process}{
%   %    \UponKW{event \postoffice.UP$\langle \data, \src, \dest \rangle$}{
%   %       $\tx \gets \langle \data, \src, \dest \rangle$\;
%   %       notify agent on \postoffice.UP(\tx)
%   %    }
%   % }
%   \Component{IPC agent}{
%     \UponKW{notification of \gw.\postoffice.UP$\langle \data, \src, \dest \rangle$ from child}{
%         $\tx' \gets$ \gw.\postoffice.UP$\langle \data, \src, \dest \rangle$\;
%         create \prf that \tx' is final at child \smr\;
%         $\tx_\textit{new}\gets\langle \tx', \prf \rangle$\;
%         submit \gw.\postoffice.\propagate($\tx_\textit{new}$) to parent \smr
%     }   
% }
% \end{algorithm}
% \subsection{Atomic Execution}
% TODO

 \section{\ipc Actors}
 \label{sec:components}

This section describes the state and functions of the two \ipc actors: the \gw and the \sa.

\subsection{\gwFull (\gw)}
\label{sec:gw}

The \gw is an actor that exists in every subnet in the \ipc hierarchy and contains all information and logic the subnet itself needs to hold in order to be part of \ipc.
The functionality of the \gw described in \Cref{sec:functionality} is summarized in \Cref{alg:gw}.
The \gw holds:
\begin{itemize}
    \item The names of its own, its parent's and its children's subnets

    \item The predicate used to evaluate the validity of \pofsFull.
    This predicate will be applied to {\pof}s from both the parent subnet and the child subnets.
    It is specific to the subnets (and the protocols they use) involved in interactions with this subnet.

    \item The \postoffice storing all the outgoing cross-net transactions, along with their routing metadata (original source and ultimate destination subnets).
    We model the \postoffice as an infinitely growing set, from which the appropriate \ipc agents select only those elements that need to be submitted to other subnets.
    A garbage-collection mechanism for deleting delivered outgoing cross-net transactions from the sender subnet's state is out of the scope of this document.
    One can imagine, however, a garbage-collection mechanism based on acknowledgments (that are themselves cross-net transactions).

    \item In a PoS-based subnet whose membership is managed by its parent, the \gw also contains the target membership that the subnet must reconfigure to (if the subnet is not using it yet).
    This membership is the subnet's local copy of the membership stored in its corresponding \saFull in the parent.
    It must be part of the subnet's replicated state, so that its replicas have a consistent view of it and can correctly reconfigure.
    Since reconfiguration does not happen immediately, the actual membership (also part of the subnet's replicated state) lags behind the target membership.

\end{itemize}

\begin{algorithm}[ht]
\footnotesize
\caption{\gwFull (\gw)}\label{alg:gw}
  \DontPrintSemicolon
  \SetKwProg{Component}{$\blacktriangleright$ \bf}{:}{\KwRet}
  \SetKwFor{UponKW}{}{}{fintq}
  \SetKw{Trigger}{trigger}
    \var{ownSubnetName}: name of the subnet the \gw resides in\;
    \var{parentSubnetName}: name of the parent subnet\;
    \var{childSubnets}: set of subnet names, initially empty\;
    \var{valid}: predicate over a \pof defining its validity criteria\;
    \var{\postoffice}: set of tuples (\var{transaction}, \var{source}, \var{destination}), initially empty\;
    \var{targetMembership}: the membership this subnet should reconfigure to if it is not yet using it (PoS only)\;
    \;
  
    \UponKW{\funcName{CreateChild}(\funcParam{name})}{
        \var{childSubnets} = \var{childSubnets} $\cup$ \{\funcParam{name}\}
    % Creates a new \sa with the given name and subnet-specific parameters (such as initial membership, etc.).
    % The subnet governed by the created \sa will be considered the child of the subnet of this \gw.
        % \replace{Creates a new \sa with the given name and subnet-specific parameters (such as initial membership, etc.).
        % The subnet governed by the created \sa will be considered the child of the subnet of this \gw.}{Registers subnet on IPC}
    }
    \UponKW{\funcName{RemoveChild}(\funcParam{name})}{
        \var{childSubnets} = \var{childSubnets} $\setminus$ \{\funcParam{name}\}
        % \matej{Is it meaningful to have this functionality at all?}%\arp{TLDR: killing a subnet is not necessarily trivial.\\ In our current implementation a client of a child subnet assumes $f<n/3$ where $n$ is the validators set at the child, but this is not necessarily a requirement of subnets. For example, state (lightning, payment) channels should be a specific case of subnets in which the set of validators = all clients of the subnet. In state channels, validators are required to sign in order to change the state (full safety but no liveness if $f\geq 1$) -- except for killing the subnet, allowing correct validators to withdraw with latest checkpoint without deadlocking their state in the corrupted subnet. This can be done with timelocks in an orderly manner. AFAIK Plasma chains even have the same functionality for clients to be able to circumvent a corrupted child subnet they belong to and skip to parent. }
    }
    \UponKW{\funcName{MintDeposited}(\funcParam{amount, account, \pof})}{
        \If{\normalfont{\var{valid}}(\pof)}{
            mint \funcParam{amount} new coins\;
            transfer minted coins to \funcParam{account}
        }
        % If \emph{\pof} is valid, adds \emph{amt} newly minted coins to account \dest.
        % A valid \pof means that a corresponding \emph{SA.Deposit(..., amt, dest)} has been successfully invoked in the parent's replicated state.
    }
    \UponKW{\funcName{Withdraw}(\funcParam{amount, account})}{
        \If{\funcParam{account}.balance $\geq$ \funcParam{amount}}{
            Burn \funcParam{amount} coins from \funcParam{account}
        }
        % Burns \emph{amount} coins from account \emph{src} to be returned to the \dest account in the parent subnet.
    }
    \UponKW{\funcName{Dispatch}(\funcParam{tx, src, dest})}{
        \var{\postoffice} = \var{\postoffice} $\cup$ \{(\funcParam{tx, src, dest})\}
        % Adds the cross-net transaction \emph{tx} to the list of transactions to be submitted to another subnet.
        % The IPC agents observing the state of this subnet will pick it up from here and perform the actual submission.
    }
    \UponKW{\funcName{Propagate}(\funcParam{tx, src, dest, \pof})}{
        \If{\normalfont{\var{valid}}(\pof)}{
            \If{\funcParam{dest} = \var{ownSubnetName}}{
                execute \funcParam{tx}
            }
            \ElseIf{$\exists s \in \var{childSubnets} \cup \{parentSubnetName\}: s$ \normalfont{is part of \funcParam{dest}}}{
                \funcName{Propagate}(\funcParam{tx, src, dest})
            }
            
        }
        % Processes an incoming cross-net transaction (subnitted by an \ipc agent) and propagates it to the next subnet.
        % If \tx{tx} is destined for the current subnet, executes it.
    }
    \UponKW{\funcName{UpdateMembership}(\funcParam{membership, \pof})}{
        \If{\normalfont{\var{valid}}(\pof)}{
            \var{targetMembership} = \funcParam{membership}
        }
        % Adds the cross-net transaction \emph{tx} to the list of transactions to be submitted to another subnet.
        % The IPC agents observing the state of this subnet will pick it up from here and perform the actual submission.
    }
 %    \add{\UponKW{Slashed($\mathcal{M}$, \pom, \pof)}{ 
 %         If \pom is a valid proof of misbehavior of a set $\mathcal{M}$ of validators, and if \pof is valid, then update state to reflect predefined punishment to misbehaviors in $\mathcal{M}$.
 % }}
\end{algorithm}

\subsection{\saFull (\sa)}
\label{sec:sa}

The \saFull (\sa) is the actor in the parent subnet's replicated state that governs a single child subnet.
It stores all information about the child subnet that the parent needs and logic that manipulates it.
The \sa is registered in the IPC hierarchy by invoking the parent's \gw.\funcName{CreateChild}(\funcParam{subnetName}) function (see \Cref{sec:create}).
The functionality of the \sa described in \Cref{sec:functionality} is summarized in \Cref{alg:sa}.
The \sa holds:
\begin{itemize}
    
    \item The predicate (\var{valid}) used to evaluate the validity of \pofsFull of the child subnet's replicated state.
    It is specific to the child subnet and the protocol it uses, and its definition is part of \funcParam{params} passed to \gw.\funcName{CreateChild} when the \sa is created.

    \item The amount of funds that are locked for use in the child subnet (\var{lockedFunds}).
    Deposits increase and withdrawals decrease this value accordingly.
    Keeping track of this value is only necessary for enforcing the firewall property, since a misbehaving child subnet might claim to withdraw more than has been deposited in it.
    Thus, before withdrawing, the \sa consults this value to make sure that the total amount of withdrawals never exceeds the amount previously deposited.

    \item Snapshots of the child subnet's replicated state obtained through invocations of the \funcName{Checkpoint} function (\var{checkpoints}).

    \item If the child subnet is a PoS-based one, the \sa also contains state required for managing the subnet's membership and the associated collaterals.
    The high-level implementation presented in \Cref{alg:sa} presents a simplified view of this state and the associated logic, as it conveys the mechanisms involved without getting lost in details.
    In particular, the presented description neglects some corner cases arising from concurrent handling of multiple staking, releasing, and/or slashing procedures.
    In a real-world implementation, however, these corner cases can easily be addressed.

    The \sa stores information on which child replica has how much collateral (\var{childMembership}),
    how much collateral (and for which replica) is staked from which account (\var{collateral}),
    and which accounts requested the withdrawal of how much collateral (\var{collateralRequests}).
    Moreover, the \sa's state contains a predicate for checking the validity of proofs of misbehavior (\var{validPom})
    and a procedure to execute when a valid PoM is received through the \funcName{Slash} function.
    
\end{itemize}


\begin{algorithm}[ht]
\footnotesize
\caption{\saFull (\sa)}\label{alg:sa}
  \DontPrintSemicolon
  \SetKwProg{Component}{$\blacktriangleright$ \bf}{:}{\KwRet}
  \SetKwFor{UponKW}{}{}{fintq}
  \SetKw{Trigger}{trigger}
    \var{valid}: predicate over a \pof defining its validity criteria\;
    \var{lockedFunds}: total amount of funds circulating in the child subnet\;
    \var{checkpoints}: set of checkpoints of the child's replicated state\;
    \var{childMembership}: map of replica identities to their respective staked collaterals\;
    \var{collateral}: map of accounts to replica identities, to staked collaterals\;
    \var{collateralRequests}: set of received but unsatisfied requests for releasing collateral\;
    \var{validPoM}: predicate over a PoM defining its validity criteria\;
    \var{slashingPolicy}: procedure to execute on reception of a valid PoM\;
    \;
  
    \UponKW{\funcName{Deposit}(\funcParam{amount, account})}{
        \var{lockedFunds} += amount\;
    }
    \UponKW{\funcName{ReleaseWithdrawn}(\funcParam{amount, account, PoF})}{
        \If{\normalfont{\var{valid}}(\pof) $\land$ \var{lockedFunds} $\geq$ \var{amount}}{
            \var{lockedFunds} $-=$ \funcParam{amount}\;
            transfer \funcParam{amount} to \funcParam{account}\;
        }
    }
    \UponKW{\funcName{Checkpoint}(\funcParam{snapshot, \pof})}{
        \If{\normalfont{\var{valid}}(\pof)}{
            \var{checkpoints} = \var{checkpoints} $\cup$ \{\funcParam{snapshot}\}\;
        }
    }
    \UponKW{\funcName{StakeCollateral}(\funcParam{account, replica, amount})}{
        \var{childMembership}[\funcParam{replica}] += \funcParam{amount}\;
        \var{collateral}[\funcParam{account}][\funcParam{replica}] += \funcParam{amount}\;
    }
    \UponKW{\funcName{RequestCollateral}(\funcParam{replica, amount, account})}{
        \If{\var{collateral}\normalfont{[}\funcParam{account}\normalfont{][}\funcParam{replica}\normalfont{]} $\geq$ \funcParam{amount}}{
            \var{childMembership}[\funcParam{replica}] $-=$ \funcParam{amount}\;
            \var{collateralRequests} = \var{collateralRequests} $\cup$ \{(\funcParam{amount, account})\}\;
        }
    }
    \UponKW{\funcName{ReleaseCollateral}(\funcParam{membership, \pof})}{
        \If{\normalfont{\var{valid}}(\pof) $\land$ \funcParam{membership} = \var{subnetMembership}}{
            \For{(\funcParam{amount, account}) $\in$ \var{collateralRequests}}{
                transfer \funcParam{amount} to \funcParam{account}\;
            }
        }
    }
    \UponKW{\funcName{Slash}(\funcParam{replica, PoM})}{
        \If{\normalfont{\var{valid}}(\funcParam{PoM})}{
            \var{slashingPolicy}(\funcParam{PoM})
        }
    }
\end{algorithm}

\section{Related Work}

\begin{itemize}

    \item ZK and optimistic rollups
    
    \item sharding
    
    \item state channels
    
    \item payment channels
    
    \item plasma channels
    
    \item PoS sidechains
    
    \item Avalanche/ICP subnets
    
    \item Polygon supernets

    \item 

    \item 

    \item 
    
\end{itemize}
 % - Implementations/templates
 %  - Different types and trade-offs of checkpoint triggers 
 %    - Periodically: time, #blocks, #withdrawn, etc.
 %    - At request: (this one is not governance funded)
 %    - Combinations of these 
 %    - Slashing functions
 %    - Atomic execution types?
 \section{IPC's reference implementation}
 \label{sec:ref-impl}
 
% \jorge{I like the idea of the section but it currently reads a little weird, particularly beyond 8.1. There is this list of components/topics and a number of entries under them, but no narrative and, for many of the entries, it's unclear whether the implementation details actually add useful details, vs. just repeating the design or listing implementation "facts" without explaining their relevance. I don't know if the solution is less content of more content, but I might start with making it less "entry"-based and more narrative, focusing on things that are either very relevant/important os potentially unexpected and hence deserving of an explanation. I think e.g. 8.4 (incentives) is written in a more useful style.}

 The reference implementation of IPC differs slightly from the description of the functionality and implementation shown in previous sections, partly due to the rapidly changing development of the system, and partly due to its concurrent implementation simultaneous with the design and improvement of the system. In this section, we describe the particular implementation choices for the reference implementation of IPC. It is not the purpose of this section to comprehensively describe the reference implementation, but to list the relevant differences of the current reference implementation compared with the description of IPC made in previous sections, as well as to list the pertinent implementation decisions of abstractions such as the \pof or the consensus protocol used by the reference implementation.

 \subsection{Preliminaries}
 The current implementation considers Filecoin~\cite{filecoin} as the rootnet and Trantor~\cite{trantor} running in child subnets, both running as the consensus layer of the Lotus blockchain client~\cite{lotus}. Filecoin is a Proof-of-Storage, longest-chain-style protocol with probabilistic finality.
 For our purposes, we note that Trantor is a BFT-style protocol that iterates through instances of PBFT \cite{castro1999practical} with immediate finality. Every decided block in Trantor contains an ordered list of decided transactions and a certificate for verification, with every $\Delta$-th block containing a checkpoint of the state. At the moment, the checkpoint being generated by Trantor is a certificate of the latest decided block provided by the Lotus client.
 
 The reference implementation makes use of IPFS-style content addressing, in that data is stored where relevant and referred to with a Multiformats-compliant~\cite{multiformats} content identifier (CID) elsewhere. In particular, CIDs that address information of a specific child's subnet can be used to retrieve the content through BitSwap~\cite{bitswap} from any of the participants running \glspl{full-node} of the subnet. This means that if a subnet only has faulty participants, the content referred to by this CID may not be available. However, this is not a problem, as IPC still preserves the subnet firewall property (\cref{sec:preliminaries}).

The reference implementation uses the Filecoin Virtual Machine (FVM) as the runtime environment in which the \gw and the \sa are deployed. Both actors are user-defined, in that any user can deploy their own modifications of the provided actors, and use them to interact with the rest of the IPC hierarchy.


\subsection{Components}
The IPC reference implementation preserves all the components described in~\cref{sec:components} without additions. We however list here implementation decisions concerning these components. The two main design decisions are $(i)$ to have one IPC agent manage the interactions across all subnets, and not one per parent-child pair, and $(ii)$ to make the \gw the entry point for all IPC functionality, with the possibility to augment the default functionality for a specific subnet with the \sa.

 \subsubsection{IPC agent} 
 \label{sec:refimplipcagent}
 In the previous sections, we considered that every parent-child pairing had an independent IPC agent process. In fact, the implementation manages to execute one single IPC agent for the entire tree of subnets that may be of relevance to a participant. This process can be executed either as a daemon or as a command-line tool. In the latter case, the IPC agent cannot participate in either checkpointing or propagating cross-net transactions. We refer to the participants running the IPC agent as a command-line tool as the IPC clients.

While a participant only runs one \ipc anget, it also runs a full node for each subnet that the participant is involved in. The IPC agent process along with all the full nodes relevant to a participant conform the participant's \emph{\gls{trust domain}}. All processes within the same trust domain assume each other's correctness.
As a result, the IPC agent can be notified of changes to the state of each of the full nodes locally run by the participant.
\subsubsection{\gw as entry point}
\label{sec:gwrefimpl}

In the reference implementation, the entry point for all functionality is the \gw, unlike in the high-level functionality described in~\cref{sec:functionality}. For any of the provided functionalities, the IPC agent submits transactions to the \gw (e.g. \gw.\funcName{Deposit}(\subnetName{C}, \funcParam{amt}, \funcParam{...})). For bottom-up transactions, the child's \gw communicates with the \sa of the child at the parent. For top-down transactions, the \gw of the parent directly communicates with the \gw at the child. Nonetheless, the IPC agent never communicates with the \sa directly, but indirectly through an \gw.

% In addition, it contains variables relevant for the functionality, namely $(i)$ the minimum required total stake per child subnet, in that a child subnet becomes inactive if it drops below the minimum (see~\cref{sec:refimplsa}; $(ii)$ the checkpoint period $\Delta$, specifying the distance (in blocks) that must be maintained between checkpoints (see~\cref{sec:refimplfunc}); and $(iii)$ CIDs of child subnets' checkpoints, that are propagated during its own checkpoints (see~\cref{sec:refimplfunc}).

     % \paragraph{Circulating supply.} 
     As a result, in the reference implementation, the \gw contains the locked funds of each subnet, i.e. the subnet's \textit{circulating supply} (unlike in \cref{sec:components}, where the \sa held the locked funds). The circulating supply of each child subnet is stored in a map at the \gw, where the key is the subnet ID. As withdrawals contain a \pof, the circulating supply suffices for the firewall property.

     % \paragraph{Cross-net transactions.} Cross-net transactions are batched together when being sent to a subnet \subnetName{S$_1$} from another subnet \subnetName{S$_2$}, with bottom-up transactions being attached to and sent with childrens' checkpoints. The \gw is in charge of executing cross-net transactions. We explain further the data structures and procedures involving cross-net transactions in~\cref{sec:cnetrefimpl}.

% The \gw is the entry point of all functionality in the reference implementation. The state of \subnetName{C}'s \gw consists of: 
% \begin{enumerate}
%     \item {\bf Basic variables and pointers.} The \gw stores a pointer to the parent's subnet, and to each of its children. 
%     \item {\bf Generic checks and implementation of functionality.} As the entry point for IPC operations, the \gw implements all functionality with calls to the subnet-specific checks and operations specified in the \sa of the child. In addition, it contains variables relevant for the functionality, namely: 
%     \begin{itemize}
%         \item the minimum required total stake per child subnet, in that a child subnet becomes inactive if it drops below the minimum;
%         \item the checkpoint period $\Delta$, specifying the distance (in blocks) that must be maintained between checkpoints (see~\cref{sec:refimplfunc}); and
%         \item CIDs of child subnets' checkpoints, that are propagated during its own checkpoints (see~\cref{sec:refimplfunc}).
%     \end{itemize}
    
%     \item{ \bf Funds and collateral.} The \gw holds the funds of all deposits to the children subnets and the stake kept as collateral of each child subnet.
%     \item {\bf State concerning the execution of cross-net transactions.} Cross-net transactions are batched together when being sent to a subnet from another subnet, with bottom-up transactions being attached to and sent with childrens' checkpoints. The \gw is in charge of executing and storing cross-net transactions. We explain further the data structures and procedures involving cross-net transactions in~\cref{sec:cnetrefimpl}.
% \end{enumerate} 

% \TODO{add sequence diagram}

% \jorge{Structure-wise, this is a bit weird. We say there are two key differences (IGA and consolidated agent), then proceed to list two decisions (and the IGA isn't the first). Then we provide an example of something (not sure what it's meant to exemplify?) that is listed in line with the implementation details. And the agent comes in its own subsection. Suggestion: let's make everything below this into subsections (or merge into existing ones). And please contextualise that example (either here or in subsection) as I really don't know what it's supposed to illustrate.}

\subsubsection{\sa for subnet customization}
\label{sec:refimplsa} In the reference implementation, \saOf{S} holds the state specific to subnet \subnetName{S}. However, the aforementioned entry point for all functionality is the \gw, and not the \sa. A subnet can augment the default functionality of the \gw in the \sa. In particular, the \sa can include conditions for the validation of proofs of finality, releases of funds and stake, and slashing rules\footnote{the reference implementation does not provide any slashing rule at the time of writing, but provides the mechanism to define slashing rules in the \sa.}.
% In particular, the state of \saOf{C} at \subnetName{P} contains: 
% \begin{enumerate}
%     \item {\bf Basic variables and pointers.} These are \subnetName{C}'s membership, a pointer to \subnetName{C}'s \gw, the parent's subnet name \subnetName{P}, a value representing the consensus mechanism used at \subnetName{C} (always Trantor at the moment), and the CID of \subnetName{C}'s genesis block (the first block of the child subnet). 
%     \item { \bf Checkpointing data and rules.} CIDs of generated checkpoints are stored in the \sa, along with \subnetName{C}-specific rules for the verification of checkpoints. We explain checkpointing in detail in~\cref{sec:refimplfunc}. 
%      \item {\bf \subnetName{C}'s finality verification.} The \sa defines the validity of a \pof attached to a state change that happened in \subnetName{C}.
%      We explain the verification of proofs of finality in~\cref{sec:cnetrefimpl}.


% \paragraph{Example.} We illustrate the interactions between the \gw as entry point and the \sa for subnet customization with an example. Let $u$ be a user with an account \accountName{a} at a subnet \subnetName{P} that implements a gaming platform. Suppose $u$ has joined a tournament at subnet \subnetName{P/C\textsubscript{T}} that has a registration cost of $c$. To do so, $u$ joins the tournament via submitting at \subnetName{P} a transaction $tx=$ \subnetName{P}.\gw.\funcName{Deposit}\funcParam{(\subnetName{P/C\textsubscript{T}}, $c$)} signed with account $a$. However, the tournament restricts registration to accounts with a rating of over $threshold$. This is a subnet-specific restriction defined at \saOf{C\textsubscript{T}}. In the reference implementation, the execution of \subnetName{P}.\gw.\funcName{Deposit}\funcParam{(\subnetName{P/C\textsubscript{T}}, $c$)} invokes a check of \subnetName{P}.\saOf{C\textsubscript{T}}.\funcName{Deposit}\funcParam{($a$, $c$)} that will succeed only if $a.rating> threshold$, and fail otherwise. If the check fails, then the top-down transaction is not propagated to \subnetName{C\textsubscript{T}} and $u$ will not be able to play the tournament. Otherwise, $u$ will be able to join by depositing the cost $c$ in $\subnetName{P}$'s \gw.

%   \marko{frankly,  I am not following this... Please re-read and try to explain being more concrete. Examples help. }\arp{Rewrote, lmk if still not clear}
%   \matej{Even after the re-write, it is still not quite clear to me. We need to be able to answer the question "What does the reader learn from this?" It feels like unless the reader already knows many details of the actual implementation (which even I don't know), things are described here for no obvious purpose (come a bit out of the blue). It could be that this section comes too early and would make more sense as a summary of what has already been described. (Same for \sa's state.)}\arp{I added pointers to where each of the items of the state are explained if not already explained, and merge the state itemize with the actual text for clarity. What are
% concrete examples of something that is not clear enough of this
% section? what do you think are other details of
% the reference implementation not already mentioned that the reader
% should know for this section?}

%  \paragraph{Reusing checkpoints' \pof.}
% Therefore, a child's transaction (or state) \tx is accepted by the parent as final by providing a \pof containing enough signatures amounting for at least $2/3$ of the voting power in the child running an instance of Trantor. The current implementation relies on collecting multiple signatures from replicas, with simple incentives to participate listed in~\cref{sec:refimplincentives}\footnote{A next step in the implementation road-map is to offer a threshold signature mechanism instead of using a multisig. For now, multisigs serve the purpose of an MVP implementation.}. This \pof is the same one used for checkpoints, as bottom-up transactions are batched there (see~\cref{sec:refimplfunc}). In other words, given transaction $tx$, the \prf($tx$) needed for bottom-up transactions is a CID to the latest block decided by the child's Trantor, which already contains a certificate to verify finality. The parent s\textbf{}ubnet considers the \prf valid if it contains signatures from replicas with at least $2/3$ of the voting power in the child. The $2/3$ bound yields the optimal resilience to Byzantine faults for the partially synchronous Trantor \TODO{references}. It also is the optimal bound for interactions between subnets in partial synchrony\TODO{references}.

\subsection{Functionality} 
\label{sec:refimplfunc}
In this section, we describe the implementation of the functionality. In the reference implementation, cross-net transactions are the cornerstone of interactions between IPC subnets. Deposits, withdrawals, staking and releasing collateral, and state changes across subnets are all implemented with cross-net transactions.

\subsubsection{Deposits}
The main difference of deposits in the reference implementation compared to the high-level description is that \saOf{C} does not hold the funds being deposited to subnet \subnetName{C}. Additionally, the reference implementation explicitly addresses incentives by requiring an IPC fee in each cross-net transaction. In particular, depositing \funcParam{amount} coins from an account \accountNameFull{P}{a} in the parent subnet \subnetName{P} to an account \accountNameFull{P/C}{b} in the child subnet \subnetName{P/C} is performed in the following steps:

\begin{enumerate}
    \item The owner of $\accountNameFull{P}{a}$ submits a transaction $tx=$ \subnetName{P.}\gw.\funcName{Deposit}(\subnetName{P/C}, \funcParam{b, amount, IPCfee}).
    \item $tx$ is ordered and executed at \subnetName{P}. The ordering and execution of $tx$ is as follows:
    \begin{itemize}
        \item \gw checks that \funcParam{IPCfee} is above a hard-coded minimum IPC base fee. The parameter \funcParam{IPCfee} is an amount of coins to be paid to the child replicas to incentivize them to participate in the validation of top-down transactions for the child\footnote{The IPC fee is different from and in addition to transaction fees for replicas to order and execute transactions in a subnet, which we mention in~\cref{sec:preliminaries} but otherwise omit in each transaction throughout the document.} (see~\cref{sec:cnetrefimpl}). 
        % \item \gw calls on \subnetName{P}.\saOf{C} to perform subnet-specific checks, should there be any,
        \item \funcParam{amount} is deposited in the \gw of the parent subnet.
        \item \gw creates a top-down transaction $tx'=$ \subnetName{P/C}.\gw.\funcName{MintDeposited}(\funcParam{b, amount, IPCfee}) and stores it in the \tqueue (see~\cref{sec:cnetrefimpl}).
        \item The top-down transaction $tx'$ is ordered and executed at the child subnet, resulting in the minting of \funcParam{amount} sent to account \accountNameFull{P/C}{b}. We detail further the ordering and execution of top-down transactions in~\cref{sec:cnetrefimpl}.
    \end{itemize}
\end{enumerate}

\subsubsection{Withdrawals}
Analogously to deposits, withdrawals must carry an IPC fee to be paid to the child replicas, and the funds to be released back at the parent subnet \subnetName{P} are being held at \subnetName{P}.\gw. Otherwise, the procedure is analogous to the one described in~\cref{sec:functionality}. More concretely, withdrawing \funcParam{amount} coins from an account \accountNameFull{P/C}{b} in the child subnet \subnetName{P/C} to an account \accountNameFull{P}{a} in the parent subnet \subnetName{P} involves the following steps:
\begin{itemize}
\item The owner of \accountNameFull{P/C}{b} submits a transaction $tx=$ \subnetName{P/C.}\gw.\funcName{Withdraw}(\funcParam{amount, a, IPCfee}).
 \item $tx$ is ordered and executed at \subnetName{P/C}. The ordering and execution of $tx$ is as follows:
 \begin{itemize}
        \item \gw checks that \funcParam{fee} is above a hard-coded minimum IPC base fee. The parameter \funcParam{IPCfee} is an amount of coins to be paid to the child replicas to incentivize them to participate in the validation of bottom-up transactions for the child (see~\cref{sec:cnetrefimpl}). 
        % \item \gw performs subnet-specific checks, should there be any.
        \item \funcParam{amount} is burned from \accountName{b}.
        \item \gw creates a bottom-up transaction $tx'=$ \subnetName{P}.\saOf{C}.\funcName{ReleaseWithdrawn}(\funcParam{amount, b, IPCfee}) and stores it in the \bqueue (see~\cref{sec:cnetrefimpl}).
        \item The bottom-up transaction $tx'$ is ordered and executed at the parent subnet. This results in \saOf{C} calling \gw to release \funcParam{amount} and send it to account \accountNameFull{P}{a}. We detail further the ordering and execution of bottom-up transactions in~\cref{sec:cnetrefimpl}.
    \end{itemize}
\end{itemize}

\subsubsection{Checkpointing} 
\label{sec:refimplcheck} A checkpoint of a subnet \subnetName{P/C} is triggered every $\Delta$ blocks decided at the child subnet. If the latest block decided meets this condition, and if the participant's full node is a replica according to the state stored at the parent, then the IPC agent starts computing the checkpoint as follows:
\begin{enumerate}
    \item The IPC agent obtains a state snapshot from the child's subnet. The state snapshot is a CID \funcParam{chkpCID} of the latest decided block of the child's subnet that contains a checkpoint certificate as \pof  (recall that every $\Delta$-th block contains a checkpoint certificate in Trantor). The \pof of the checkpoint is the certificate of the block.
    \item The IPC agent obtains the CIDs of all new grandchildren's checkpoints stored at \subnetName{P/C}.\gw.\dataField{gcChkps} and of the bottom-up transactions in the \bqueue \dataField{BUpTxs} (see~\cref{sec:cnetrefimpl}). Explicitly checkpointing children checkpoints recursively bubbles up the security anchor from lower levels of the hierarchy.
    \item The IPC agent submits $tx=$\subnetName{P}.\saOf{C}.Checkpoint(\subnetName{P/C}, \funcParam{chkpCID, gcChkps, BUpTxs}, \pof).
    \item The \saOf{C} verifies the validity of the \pof, saves the checkpoint CID in its state and calls on \gw to save all checkpoints' CIDs (those of \subnetName{P/C}'s children and of \subnetName{P/C}) and to execute all bottom-up transactions attached to the checkpoint. If the \pof is not valid according to the state at \saOf{C}, then the entire checkpoint will fail.
\end{enumerate}
% \begin{algorithm}[H]
% \footnotesize
% \caption{Checkpoints \impl \label{alg:chkpsimpl}}
%   \DontPrintSemicolon
%   \SetKwFunction{FMain}{Global}
%   \SetKwProg{Pn}{Function}{:}{\KwRet}
%   \SetKwInOut{Input}{input}
%   \SetKwProg{Component}{$\blacktriangleright$ \bf}{:}{\KwRet}
%   \SetKwFor{UponKW}{upon}{do}{fintq}
%    \Component{IPC agent}{
%    \UponKW{\dataField{newBlock} \text{from subnet \subnetName{C}}}{
%         \If{\dataField{newBlock}.\dataField{blockheight} $\bmod$ $\Delta$ $=0$}{
%             \If{\subnetName{P}.\saOf{C}.\funcName{isReplica}\funcParam{(Self)}}{
%                 \dataField{chkpCID} $\gets$ \dataField{newBlock}.\funcName{GetCID()}\;
%                 \dataField{gcChkps} $\gets$ \subnetName{C}.\gw.NewChkps() \tcp*[r]{grandchildren's chkps}
%                 \dataField{BUpTxs} $\gets$ \subnetName{C}.\gw.\dataField{BottomUpRegistry}\;
%                 submit \subnetName{P}.\gw.\funcName{Checkpoint(\subnetName{P/C}, \dataField{chkpCID, gcChkps, BUpTxs})}\;
%             }
%         }
%         }
%   }
%     \Component{\subnetName{P}.\gw.\funcName{Checkpoint}\funcParam{(\subnetName{P/C}, chkpCID, gcChkps, BUpTxs)}}{
%   \If{\subnetName{P}.\saOf{C}.\funcName{Checkpoint}\funcParam{(chkpCID, gcChkps, BUpTxs)}}{
%     Save \funcParam{chkp} in the state\;  
%     Execute \dataField{tx},  $\forall$ \dataField{tx} $\in$ \funcParam{chkp.\dataField{BUpTxs}}
%     Save \funcParam{chkp} in the state\;
%   }  
%   }
%   \Component{\subnetName{P}.\saOf{C}.\funcName{Checkpoint}\funcParam{(chkpCID, gcChkps, BUpTxs)}}{
%   \tcp*[l]{[subnet-specific treatment of checkpoints and bottom-up txs]}
%     \textbf{return} \subnetName{P}.\saOf{C}.\funcName{verify}\funcParam{(chkpCID, gcChkps, BUpTxs)}\tcp*[r]{\pof is \texttt{newBlock}'s certificate}
%     % \subnetName{P}.\gw.\funcName{Checkpoint}(\funcParam{P/C, chkp})
%     }
%   % \Component{\subnetName{P}.\gw.\funcName{Checkpoint}\funcParam{(P/C, chkp)}}{
%   % Save \funcParam{chkp} in the state\;
%   % Execute \dataField{tx},  $\forall$ \dataField{tx} $\in$ \funcParam{chkp.\dataField{BUpTxs}}
%   % }
  
% \end{algorithm}

\subsubsection{Propagating cross-net transactions}
\label{sec:cnetrefimpl} 
Similarly to the description in~\cref{sec:functionality}, the current reference implementation uses the \postoffice in each \gw to propagate cross-net transactions to a subnet not immediately adjacent by submitting multiple cross-net transactions in the parent-child hierarchy (one in each subnet along the path to the destination subnet). As shown in~\cref{sec:functionality}, a cross-net transaction $tx$ is propagated to each immediately adjacent subnet along the path to its destination by traversing through the \postoffice of all intermediate subnets, via cross-net transactions $tx'(tx)$ containing $tx$ as payload. 

However, once $tx'$ is ordered and executed at an intermediate subnet \subnetName{S}, an IPC agent must pay for the cost to pay the fees for ordering and executing another transaction $tx''(tx)$ in \subnetName{S} so as to move the state to the next subnet along the path. For this reason, the reference implementation leaves $tx$ in the \postoffice of that intermediate subnet until the account that originally triggered the cross-net transaction creates $tx''$ and pays for the fees required to execute it\footnote{a whitelist of accounts that are allowed to create and pay can be provided.}. This process is repeated until $tx$ reaches its destination subnet.

 As a result, a cross-net transaction that has been paid for leaves the \postoffice to join a FIFO queue, known as either \emph{the \bqueue} or \emph{\tqueue}. All three, \postoffice, \bqueue and \tqueue, contain cross-net transactions and are part of the state of \gw, but only those transactions in either \tqueue or \bqueue are propagated. Transactions in the \tqueue (resp. \bqueue) at \subnetName{P}.\gw are top-down (resp. bottom-up) transactions.

% More concretely, suppose an account \accountName{a} submits a transaction in a child subnet \subnetName{G/P$_1$/C} that triggers a state change in its uncle subnet \subnetName{G/P$_2$}. Then, the cross-net transaction $tx$ that represents that state change must reach the grandparent through the intermediate parent subnet \subnetName{P$_1$}. In the reference implementation, this means that first a bottom-up transaction $tx_{b_1}(tx)$ with $tx$ as payload, \subnetName{C} as source and \subnetName{P$_1$} as recipient is submitted to \subnetName{P$_1$}'s \postoffice from \subnetName{C}'s \bqueue. Once $tx_{b_1}$ is ordered and executed at \subnetName{P$_1$}, \accountName{a} needs to pay for the cost of moving $tx$ into \subnetName{P${_1}$}'s \bqueue, creating a new transaction $tx_{b_2}(tx)$ with $tx$ as payload, \subnetName{P${_1}$} as source and \subnetName{G} as recipient. Following, $tx_{b_2}$ is ordered and executed at \subnetName{G}, reaching the \postoffice at \subnetName{G}. Finally, once \accountName{a} pays to create a transaction $tx_{t_1}(tx)$ with $tx$ as payload, \subnetName{G} as source and \subnetName{P$_2$} as recipient, can $tx_{t_1}$ be ordered an executed, meaning that $tx$ reaches the \postoffice at its destination, and will be executed once one last account pays for its ordering and execution at \subnetName{P$_2$}. 

% \paragraph{Nonce.} Once a subnet orders and executes cross-net transactions, the \gw updates its state by increasing a nonce specific for the recipient subnet (i.e. the parent if bottom-up or the specific child if top-down). The nonce is necessary to avoid replay attacks. We detail now the particularities of top-down and bottom-up transactions. 

\paragraph{Top-down transactions.}

 In order to prevent inconsistencies across replicas, the IPC agent does not immediately submit a top-down transaction to the child subnet. Instead, \ipc agents consistently broadcast the top-down transactions they consider as final at the parent subnet. Also, as an optimization, the IPC agent batches top-down transactions in a \emph{\tcheckpoint} that is consistently broadcast to other replicas every $\Delta_T$ blocks. In this consistent broadcast, an IPC agent broadcasts batches of top-down transactions that it locally considers as valid, and it in turn signs a received batch if it considers all transactions of the batch as valid. As such, the $\pof$ of a top-down transaction $tx$ is obtained once enough signatures to form a certificate\footnote{In Trantor, a certificate for the batch consists of at least enough replicas containing a supermajority of the voting power sign the batch.} are received for a batch containing $tx$.

A participant running a straggling parent full node that receives a certificate for a batch as \prf, but that does not locally see all transactions of the batch as valid, can instead verify the \prf. Once the IPC agent verifies a certificate for a batch of transactions, the IPC agent submits the batch to the child subnet for ordering and execution. 

% This way, when a supermajority of correct child replicas locally see and consider the corresponding transaction at the parent as final in their parent subnet, the rest of the replicas can instead verify the certificate to update the state of the child subnet, preventing inconsistencies with participants running straggling parent replicas. 
    % \del{ In order to consider a transaction \tx at the parent as final, we use the fact that a participant has view into a version of the parent subnet (through its local parent replica process) \jorge{not sure what this means}. In this case, the \prf contains the block height~$h$ (and pointer to that block) at the parent subnet. A child replica then asserts in its parent that the state is final by checking its local version of the parent blockchain at height~$h$. If the local version at the parent replica did not reach height~$h$ yet, the child replica considers the state to currently be non-final/non-valid. The child replica checks again when the parent replica reaches height~$h$.}

\paragraph{Bottom-up transactions.}  
The child subnet aggregates bottom-up transactions attaching their corresponding CIDs to the next checkpoint, along with an increasing nonce value per CID that is unique for each parent-child pair\footnote{the nonce value is necessary to prevent replay attacks}. The CIDs of these bottom-up transactions are placed in the \gw of the child. Bottom-up transactions are stored in the \gw of the child until it is time to checkpoint to the parent (see~\cref{sec:refimplcheck}). This way, the \gw serves as the single location for the CIDs of bottom-up transactions and the IPC agent only needs to monitor the \gw to get all necessary information from the child subnet. As shown in~\cref{sec:refimplcheck}, since the CIDs of bottom-up transactions are attached to the checkpoint, the propagation of bottom-up transactions depends on the validity and execution of the checkpoint transaction. 

% Although omitted in~\cref{alg:chkpsimpl}, executing each \dataField{tx} in \funcParam{chkp.}\dataField{BUpTxs} may require additional checks by the \sa. An example is a withdrawal of funds or a release of collateral for a replica leaving the subnet, that will require a call to check validity of the withdrawal transaction by calling the subnet's \saOf{C}. Recall that checkpoint transactions are only atomically executed if all transactions of the batch are valid and are executed as well.

\subsubsection{Creating and removing a child subnet} Analogously to~\cref{sec:functionality}, subnets are created by instantiating a new \sa and registering the \sa in \gw. When the \gw contains a minimum amount of collateral stored associated with \sa (where enough is defined in the \gw), the subnet can be registered in \gw. 
% This registration in the \gw is what allows this subnet to interact with the rest of subnets registered in IPC through \gw, and thus we refer to it as the creation of the subnet. 

A subnet \subnetName{P/C} is removed from IPC in three steps. First, all users must withdraw their funds to set the circulating supply to zero. Second, all validators leave and release their collateral. Third, any account sends a \subnetName{P}.\gw.\funcName{kill}(\funcParam{\subnetName{C}}) transaction to the \gw that marks the subnet as removed.

\subsubsection{Staking and releasing collateral}
Contrary to~\cref{sec:functionality}, staking collateral does not require a explicit call to update any state in the \gw of the child subnet. A replica simply increases its stake with account \accountNameFull{P}{a} by submitting a transaction \tx{tx} = \funcNameFull{P}{\gw}{StakeCollateral}(\accountNameFull{P}{a}, \funcParam{replica}, \funcParam{amount}). This transaction stakes the collateral in the \gw of the parent \subnetName{P}. Notwithstanding, it does not incur a reconfiguration in the weights of the replica set running the Trantor protocol at the child subnet \subnetName{P/C}. Instead, it is up to the current replica set at \subnetName{P/C} to decide when and how to update their membership to reflect this change of the relative voting power, and notify the new reconfiguration to \subnetName{P}.\gw with a bottom-up transaction. 

A reconfiguration transaction \tx{tx} = \funcNameFull{P}{\saOf{C}}{UpdateMembership}(\funcParam{membership}, \pof) being submitted to the parent can also trigger a release of collateral \tx{tx} = \funcNameFull{P}{\gw}{ReleaseCollateral}(\funcParam{membership}) to reflect this update in the weighted voting of the child subnet. It is possible that a reconfiguration triggers a release of collateral that drops the staked collateral below the minimum threshold for subnet creation. IPC subnets of the reference implementation must always hold this minimum amount of collateral per subnet (defined in the \gw). If, after it has been created and registered, the subnet's collateral drops below the required minimum, then the subnet enters an \emph{inactive} state. This means that the subnet can no longer interact with the rest of the active subnets registered in the \gw, or even withdraw funds back to the parent subnet. In this case, though, users and remaining replicas can stake enough collateral to reactivate the subnet. 
\subsection{Incentives} 
\label{sec:refimplincentives}
In the current reference implementation, replicas get rewarded for executing the checkpoint algorithm and participating in the \tcheckpoint by charging an IPC fee on all cross-net transactions. This incentivizes replicas to participate, even if that costs them a fee to be paid for the transaction at the parent. All cross-net transactions must contain a fixed amount known as IPC fee (on top of the standard transaction fee required for ordering and execution of the transaction in the corresponding subnet). The IPC fee is only paid once the checkpoint (or \tcheckpoint) is ordered and executed. However, no specific incentives are given at the moment to the submitter(s) of the checkpoint transaction, or to the signers of the certificate, in that the sum of IPC fees of the batch is evenly distributed proportionally to the stake of each replica in the membership. This can lead to equilibria in which some participants are incentivized neither to submit checkpoints nor participate in generating a 
\pof. We are currently working in more complex incentive-compatible mechanisms that ensure rational participants will follow the protocol for the reference implementation, via rewards and slashing.
% \del{ The \impl makes certain implementation decisions that affect the design of checkpoints, particularly:
% \begin{enumerate}
% \item The \gw stores the checkpoints from all subnets, instead of having each subnet store their respective checkpoints at their specific \sa.
% \item Checkpoints are triggered periodically after $\Delta$ blocks are decided at the child
% \item cross-net transactions are batched together with checkpoints and forwarded to the parent following checkpoint triggers.
% \end{enumerate}
% }

% \del{\TODO{Discuss/verify} These implementation decisions come with a number of advantages. First, parent and child subnets periodically synchronize following predictable events. Second, having the \gw store all checkpoints easily enables batching of checkpoint transactions with cross-net transactions. Third, this batching inherently provides participation incentives for replicas to submit checkpoints, as they will be rewarded with the IPC fee from cross-net transactions.\guy{I would say that the third point is a problem not an advantage. It entangles checkpointing with cross-net transactions and might make checkpoints depend on the cross-net transactions for incentivation.} \arp{Agreed, but iiuc we will not have other incentives by M2/M3 for checkpoints, so some incentives are better than none at all for checkpoints. But yes, down the road we should work on independent checkpoint incentives (governance account).}}

\section{Conclusion}
The scalability challenge posed by the need of consensus in blockchain
networks has led to the exploration of L2 and subsequent layers. While there has been an
explosion of L2 solutions in recent years, most of these lack of a non-monolithic design that allows for
native communication across state partitions, on-demand horizontal
scaling, decentralization, and recursive security anchoring with the
root layer. We presented in this document Interplanetary Consensus (IPC), a blockchain architecture based on web-scale on-demand horizontal
scalability through a hierarchical subnet structure. IPC leverages the
security of parent subnets to benefit child subnets and allows subnets
to be used in different ways, such as hosting different applications
or sharding a single application, or even user-motivated state partitions that are not necessarily immediate at the time of writing. As a result, we believe IPC represents a significant step
forward in the quest for L2+ scaling, providing for a sensible, flexible design that balances decentralization,
scalability, security, and usability at relatively short timescales.
 
% ----------------------------------------------------------------
% ----------------------------------------------------------------

%%
%% The next two lines define the bibliography style to be used, and
%% the bibliography file.
\bibliographystyle{plain}
\bibliography{bibliography}

%%
%% If your work has an appendix, this is the place to put it.
\appendix
% ----------------------------------------------------------------
% ----------------------------------------------------------------
\newpage
\printglossaries
% \section{Glossary}
\label{sec:gls}
\TODO{(Marko)Add IPC Glossary~\cite{glossary} here and move most of model down here}


% \input{sections/algorithm.tex}
% \input{sections/QvW.tex}
% \input{sections/Yao.tex}
% ----------------------------------------------------------------
% ----------------------------------------------------------------

%%%%%%%%%%%%%%%%%%%%%%%%%%%%%%%%%%%%%%%%%%%%%%%%%%%%%%%%%%%%%%%%%%
%%%%%%%%%%%%%%%%%%%%%%%%%%%%%%%%%%%%%%%%%%%%%%%%%%%%%%%%%%%%%%%%%%

\end{document}
\endinput
%%
%% End of file 
