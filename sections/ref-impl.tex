 % - Implementations/templates
 %  - Different types and trade-offs of checkpoint triggers 
 %    - Periodically: time, #blocks, #withdrawn, etc.
 %    - At request: (this one is not governance funded)
 %    - Combinations of these 
 %    - Slashing functions
 %    - Atomic execution types?
 \section{IPC's reference implementation}
 \label{sec:ref-impl}
 
 In this section, we describe the particular choices implemented by the Consensus Lab team for the reference implementation of IPC.  
 The current implementation considers Filecoin as the rootnet, and Trantor running in child subnets. For our purposes, it is important to note that Trantor is a BFT consensus protocol with immediate finality, and Filecoin is a longest chain style protocol with probabilistic finality. 

\subsection{Components}
The IPC reference implementation preserves all the components described in~\cref{sec:components} without any additions. We however list here implementation decisions concerning these components.

\paragraph{\gw and \sa.} The \gw is the only built-in actor of the reference implementation. For this reason, it is the \gw that receives the funds to be deposited for all subnets, and that releases the funds on withdrawals. It also holds the collateral of members of the validator's list. In contrast, the \sa is user defined by the creators of that subnet, and thus it holds the subnet-specific information, such as the consensus mechanism used, the checkpointing data and rules, etc. \TODO{Check state on github's repo and update this paragraph}

\paragraph{Compressed accounting.}
In \cref{sec:components} we mentioned that the state of \sa contains accounting data. In the reference implementation, \sa contains as accounting data of the subnet a single variable representing the sum of all the individually locked funds at the parent which are dedicated to the child subnet, i.e. the subnet's \textit{circulating supply}. 

\paragraph{Content addressing.} The reference implementation makes use of IPFS-style content addressing, in that data is stored where relevant and referred to with a Multiformats-compliant content identifier (CID) elsewhere. In particular, CIDs that refer to information of a specific child's subnet can be retrieved through BitSwap from any of the subnet's replicas. We assume thus that this is information is available (that is, there is at least one correct replica in all subnets). \arp{Question: what is the worst that can happen if this is not the case?? Ideally: only this subnet not being able to properly resolve transactions involving itself}

\paragraph{IPC agent and metadata.} In the previous sections, we considered that every parent-child pairing will have an independent IPC agent process. In fact, the implementation manages to execute one single IPC agent for the entire tree of subnets that may be of relevance to the participant. This process can be executed either as a daemon or as a command-line tool. In the latter case, the IPC agent cannot participate in either checkpointing or propagating cross-net transactions\arp{basically a user using IPC wallet once we separate both}. 


\subsection{Topdown messages}
Once the parent subnet orders and executes the parent transactions corresponding to topdown messages, the \gw updates its state by increasing a nonce specific for the child. the IPC agent adds the respective child transactions to the cross-net messages pool (referencing the same nonce). Then, the IPC agent generates a \pof by running an instance of an all-to-all broadcast that generates signed \emph{certificates} containing a supermajority of signatures from child validators\footnote{\add{an example of such a protocol can be found in Trantor's availability module.}}. In this all-to-all broadcast, validators broadcast the topdown messages that they locally consider as valid (as participants run a local parent replica and the IPC agent is notified of changes to the local replica). This way, when a supermajority of correct child validators see and consider the corresponding transaction at the parent as final, the rest of the validators will not have to wait for them to locally see the corresponding state at their parent replica, but can instead verify the certificate, preventing inconsistencies. After the IPC agent generates (or verifies) this \prf, the IPC agent provides the transactions to the child replica for ordering and execution.\arp{In reality, atm the way this actually works is with local validity check after execution at the child replica, which can generate inconsistencies between updated and straggling replicas, but there's works to change this to what is described aboved: \url{https://github.com/consensus-shipyard/ipc-agent/issues/72}}

    % \del{ In order to consider a transaction \tx at the parent as final, we use the fact that a participant has view into a version of the parent subnet (through its local parent replica process) \jorge{not sure what this means}. In this case, the \prf contains the block height~$h$ (and pointer to that block) at the parent subnet. A child replica then asserts in its parent that the state is final by checking its local version of the parent blockchain at height~$h$. If the local version at the parent replica did not reach height~$h$ yet, the child replica considers the state to currently be non-final/non-valid. The child replica checks again when the parent replica reaches height~$h$.}

\subsection{Bottomup messages}  
\TODO{Still a few changes needed after chat with Alfonso}
\paragraph{Batching through the \gw.} The child subnet aggregates the cross-net messages from within its own subnet and those propagated from its children, and includes their CIDs in the next checkpoint. All related bottomup transactions are made via the \gw of the child. Bottomup transactions are batched there until it is time to checkpoint at the parent (specifically, every $\Delta$ child blocks). This way, the \gw serves as the single data structure (queue) storing bottomup transactions and the IPC agent only needs to monitor a single location to get all necessary information from the child subnet. Since \postoffice transactions are included in the batch, the execution of the batch also depends on the \postoffice functionality at the parent handling those transactions (recall that the parent's \postoffice is located at the parent's  \gw).

 \paragraph{Checking for Finality.}
Therefore, a child's transaction (or state) \tx is accepted at the parent as final by providing a \prf containing enough signatures amounting for at least $2/3$ of the voting power in the child running an instance of Trantor\footnote{The current implementation relies on collecting multiple signatures. A next step in the implementation road-map is to offer a threshold signature mechanism instead of using a multisig. For now, multisigs serve the purpose of an MVP implementation.}.
 
 In other words, given that \tx is a CID, the \prf($tx$) needed for bottomup messages is a CID to the latest block decided by the child's Trantor consensus protocol, which already contains a certificate to verify finality. The parent subnet considers the \prf valid if it contains signatures from validators with at least $2/3$ of the voting power in the child. The voting power is measured according to what is stored in \sa for the epoch containing \tx. \arp{To verify: is membership stored at SA if SA is user-defined? or at \gw as well?}


\paragraph{Checkpointing.} We show in~\cref{alg:chkpsimpl} the main design choices made by the reference implementation. A checkpoint is first triggered every $\Delta$ blocks decided at the child subnet. If the latest block decided meets this condition, and if the participant's child subnet is a validator according to the state stored at the parent, then the IPC agent starts computing the checkpoint as follows:
\begin{itemize}
    \item The IPC agent obtains a state snapshot from the child's subnet.
    \item The IPC agent obtains the CIDs of all new grandchildren's checkpoints, and of the bottomup \postoffice messages
    \item The IPC agent computes the checkpoint $chkp$, compressing its state by computing only the additional changes from the latest checkpoint stored at the parent's SA, and creates a \pof(chkp)
    \item The IPC agent submits $tx=P.\sa.Checkpoint(chkp, \pof(chkp))$
    \item The parent subnet, upon ordering and executing $tx$, saves $chkp$ in the state. \TODO{Saves a CID after submitting it to IPFS? or the actual checkpoint?} \TODO{and updates also \gw with the new postbox messages?}
\end{itemize}

\begin{algorithm}[H]
\footnotesize
\caption{Checkpoints \impl}\label{alg:chkpsimpl} \TODO{Update to reflect CIDs etc.}
  \DontPrintSemicolon
  \SetKwFunction{FMain}{Global}
  \SetKwProg{Pn}{Function}{:}{\KwRet}
  \SetKwInOut{Input}{input}
  \SetKwProg{Component}{$\blacktriangleright$ \bf}{:}{\KwRet}
  \SetKwFor{UponKW}{upon}{do}{fintq}
   \Component{IPC agent}{
   \UponKW{newBlock \text{from child subnet}}{
        \If{\texttt{newBlock}.blockheight $\bmod$ $\Delta$ $=0$}{
            \If{P.SA.isValidator(Self)}{
                \textit{state} $\gets$ obtain state snapshot from child;\;
                \textit{gcChkps} $\gets$ query child's \gw for grandchildren's checkpoints \;
                \textit{postboxmsgs} $\gets$ query child's \gw for bottomup postbox messages\; 
                \textit{pChkps} $\gets$ query parent's \gw for previous checkpoints  \; 
                \textit{chkp} $\gets$ \textit{createChkp}(\text{state}, \textit{pChkps.latestChkp}, gcChkps, postboxmsgs) \tcp*[r]{compress state}
                create $\pof(chkp)$ \tcp*[r]{multisig}
                submit $P.\sa.Checkpoint(chkp, \pof(chkp))$\;
            }
        }
        }
  }
  
  \Component{P.\sa.Checkpoint(chkp, \pof(chkp))}{
    verify($\pof(tx')$)\;
    save $chkp$ in the state\;
    \TODO{Call \gw to store there the bottomup postbox msgs}
    }
  
\end{algorithm}

\subsection{Other operations}
\paragraph{Create.} Subnets are created by instantiating a new \sa and registering the \sa in \gw. When the \gw contains a minimum amount of collateral for \sa (where enough is user defined in the \sa \arp{Question: is it?}), the subnet can be registered in \gw. This registration in the \gw is what allows this subnet to interact to the rest of subnets registered in \gw through IPC, and thus we refer to it as the creation of the subnet. 
\paragraph{Join and leave.} Validators join the validator set by depositing the required collateral in the \gw, collateral that they recover by leaving the subnet (provided they have not been slashed before). If the amount of collateral drops below the required minimum (e.g. by a validator, or validators, leaving the subnet), then the subnet enters an \emph{inactive} state. This means that the subnet can no longer interact with the rest of the active subnets registered in the \gw, until the minimum collateral is restored via deposits from validators.
\paragraph{Removing a subnet and retrieving funds.} An active subnet can be removed by a majority of current validators, releasing all the collateral and the circulating supply back at the parent. This prevents validators and users from having their funds stuck at the \gw once enough validators recover their collateral and leave the subnet inactive. In this case, though, users and remaining validators can retrieve their funds by either (i) depositing enough funds as collateral as needed to reactivate the subnet; or (ii) retrieving their funds thanks to the latest snapshot checkpointed at the parent. 
\paragraph{Slashing.}\arp{Nothing atm, should we even mention it then in previous sections? Option2: describe here some of the sketches in issues of ipc-agent (ipc-actors?) repos (“next” label)}

\subsection{Incentives} 
At the moment, validators get rewarded for executing the checkpoint algorithm by charging an IPC fee on all transactions traversing the postbox. This incentivizes validators in participating on the checkpointing functionality, even if that costs them a fee to be paid for the transaction at the parent. \jorge{in a given checkpointing period, which validators charge the fees, which validators pay for the checkpoint, and are these the same? in other words, are incentives tightly aligned?}
\TODO{TBC. No governance account, but \textbf{participation incentives} by: 
    \begin{enumerate}
        \item Additional IPC fee for validators to relay
        \item Incentivizing checkpoint submission as a result of batching with other cross-net messages (which contain the IPC fees), further justifying checkpoints being stored at \gw.
        \item Keeping Stake at SA and slashing through fraud proofs
    \end{enumerate} 
}

\TODO{Deposits, withdrawals, check collateral conditions? something else worth mentioning? How are subnets created in the reference impl, and killed? how does one join/leave (something worth mentioning)?}
\TODO{When is collateral checked, and what happens if collateral not met?}
\TODO{Nodes are just rewarded with IPC hardcoded base fees. Are they ever punished? how? (lost collateral by equivocation, checkpoint omission?, removed from replica set?}
% \del{ The \impl makes certain implementation decisions that affect the design of checkpoints, particularly:
% \begin{enumerate}
% \item The \gw stores the checkpoints from all subnets, instead of having each subnet store their respective checkpoints at their specific \sa.
% \item Checkpoints are triggered periodically after $\Delta$ blocks are decided at the child
% \item cross-net messages are batched together with checkpoints and forwarded to the parent following checkpoint triggers.
% \end{enumerate}
% }

% \del{\TODO{Discuss/verify} These implementation decisions come with a number of advantages. First, parent and child subnets periodically synchronize following predictable events. Second, having the \gw store all checkpoints easily enables batching of checkpoint messages with cross-net messages. Third, this batching inherently provides participation incentives for validators to submit checkpoints, as they will be rewarded with the IPC fee from cross-net messages.\guy{I would say that the third point is a problem not an advantage. It entangles checkpointing with cross-net messages and might make checkpoints depend on the cross-net messages for incentivation.} \arp{Agreed, but iiuc we will not have other incentives by M2/M3 for checkpoints, so some incentives are better than none at all for checkpoints. But yes, down the road we should work on independent checkpoint incentives (governance account).}}
