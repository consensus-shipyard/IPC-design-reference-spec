\section{Miscellaneous}
\label{sec:misc}

\subsection{Shortcuts as a Service}
\red{Q: Can IBC be implemented independently as a ``shortcut" between subnets in the ``simple" model?}\\
A general IBC channel between subnets might be problematic since it would damage the subnet isolation guarantees. 
However, a service providing a ``shortcut" can be build as an application over the simple model, in a similar manner to bidirectional payment channels.
A service provider opens 2 accounts -- one in each subnet -- and deposits tokens in each. Denote these subnets by $\SN_A$ and $\SN_B$, and the service accounts as $s_A\in\SN_A$ and $s_B\in\SN_B$.
Now say that account $a$ from subnet~$A$ wants to transfer $m$~tokens to account $b\in\SN_B$. 
It can do so by transferring $m$ tokens in $\SN_A$ to $s_A$ and having $s_B$ transfer $m$ tokens in $\SN_B$ to $b$. Thereby creating a ``shortcut" instead of traversing the hierarchy tree.
Clearly, for this service~$a$ must pay fees to the service provider (to~$s_A$).

\arp{this is HTLCs}The trust/risk in this service falls only on the ones who participate in it, namely, the users behind~$a$ and~$b$, and the service provider.
Smart contracts can govern this service (instead of trusting the service provider).
Essentially, $a$ deposits the tokens in the smart contract which releases them to the control of $s_A$ given a proof that $b$ have received the tokens from~$s_B$.
The proof is based on $\SN_B$ guarantees, for example, the transaction appearing in $\SN_B$'s blockchain at a given depth.
The tokens would be released back to~$a$ in case a timer expired without~$s_A$ providing the proof.
To reduce costs in the common case, I suggest to include an ``cooparation'' release mechanism.
By adding the possibility for~$a$ to approve that the transaction happened to its satisfaction, the need to include a proof from~$\SN_B$ in~$\SN_A$'s blockchain is mitigated, when everyone is honest.
To encourage the use of the cooperation release by~$a$, the service provider can offer reduced fees whenever the release happens due to~$a$'a approval.

While this service entails liquidity costs (for the service provider), having it as an application on top of the HC framework provides several benefits.
(1) Service providers are expected to create shortcuts where they are most profitable, which should typically correlate with where they are most needed.
(2) Competition may arise among different service providers, driving for better service and reduced costs.
(3) All is done without central intervention.

% This service could easily be generalized to include multiple subnets. A service provider can have accounts in multiple subnets, thereby averaging the flow among many channels.
