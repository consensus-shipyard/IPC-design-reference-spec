\section{IPC parent-child interactions}
\label{sec:functionality}

We now focus on the interaction between two subnets in a parent-child relation, which is the basic building block of the recursive \ipc hierarchy.
The IPC interface exposes the following functionalities:
\begin{enumerate}

    \item Creating child subnets in the IPC hierarchy.\\
    \emph{E.g., when starting a new game on the gaming platform.}
    
    \item Depositing funds from an account in a subnet to an account in its child.\\
    \emph{E.g., when a player tops up the balance of their account on the gaming platform.}
    
    \item Withdrawing funds from an account in a subnet to an account in its parent.\\
    \emph{E.g., when a player withdraws money they won in a tournament.}
    
    \item Checkpointing a subnet's replicated state in the replicated state of its parent.\\
    \emph{E.g., after every 100 blocks of transactions applied to the gaming platform's subnet.}
    
    \item Invoking actor functions across subnets, i.e., the replicated logic of one subnet acting as a client of another subnet.\\
    \emph{E.g., when a game finishes and players' rankings are automatically updated.}
    
    \item Removing child subnets from the IPC hierarchy.\\
    \emph{E.g., when a game fihishes, rankings have been updated, and the state of the game can be disposed of.}
    
    \item Managing proof-of-stake subnets, exposing functions for adding and removing replicas, managing the associated collateral, and slashing of provably misbehaving replicas.\\
    \emph{E.g., when mutually distrusting players play together.}
\end{enumerate}
In the following, we describe each functionality in detail, introducing the functions of the \gw and \sa through which this functionality is exposed
and the patterns in which the users and the \ipc agent invoke them via transactions.

\subsection{Creating a child subnet}
\label{sec:create}

To create child subnets, the \gw exposes the following function.
\begin{align*}
    \gw.&\funcName{CreateChild}(\funcParam{subnetName})
\end{align*}
Any user or actor of a subnet \subnetName{P} can create a new child subnet \subnetName{P/C} by
\begin{enumerate}
    \item creating a new instance of the \saFull \saOf{C} and
    \item submitting a transaction \funcNameFull{P}{\gw}{CreateChild}(\subnetName{C}).
\end{enumerate}

The new actor \saOf{C} must be configured with all the subnet-specific parameters relevant for governing the new subnet.
These would usually include the used consensus protocol, rules for joining the subnet, definitions and evaluation logic for {\pom}s and {\pof}s, and slashing policies.
From the perspective of the \ipc hierarchy, the subnet is considered created as soon as \saOf{C} is created.
The subnet itself need not necessarily be operational at this moment,
as the parent subnet always has a passive role when it comes to interacting with it.

\begin{example}
\label{ex:create-game-subnet}

Imagine that a player wants to create a game subnet to play against 3 opponents,
such that each player will run their own replica in the game subnet (i.e., their own copy of the game server).
As an incentive for honest behavior, the player decides that the child subnet will be PoS-based (see \Cref{sec:pos-subnets}),
where each replica must be backed by a minimal collateral of 10 coins that would be slashed (and, say, redistributed to the other players) if the replica is caught misbehaving (see \Cref{sec:slash}).

To achieve this, the player creates a subnet actor governing a child subnet that allows a replica to join only if at least 10 coins of collateral are associated with it, and 
stops accepting new collateral after 4 replicas (the player and 3 opponents) reach the threshold of 10 coins.
The player then registers the new subnet through a \funcNameFull{P}{\gw}{CreateChild} transaction, starts their own replica of the game server (we assume the game server is implemented such that it can be replicated and run as a subnet)
and waits for other players's replicas to join (see \Cref{sec:staking-collateral}).

Note that, since the subnet actor is created by the user, its initial state and logic can be configured arbitrarily.
For example, the \sa could easily be configured with other admission policies (only players with a game ranking within a defined range)
or slashing policies (penalize misbehaving replicas by the backing player loosing not just coins, but also their position in a game-specific ranking system).

\end{example}

\subsection{Depositing funds}
\label{sec:deposit}

A deposit is a transfer of funds from an account in the parent subnet to an account in the child subnet.
The following functions are exposed by the \ipc actors to enable deposits.
\begin{align*}
    \sa.&\funcName{Deposit}(\funcParam{amount, account})\\
    \gw.&\funcName{MintDeposited}(\funcParam{amount, account, \pof})
\end{align*}
The \funcParam{amount} is the amount of funds to be deposited, \funcParam{account} is the destination account in the child subnet, and \funcParam{\pof} is a proof of finality proving that \sa.\funcName{Deposit}(\funcParam{amount, account}) has been applied to the parent subnet's replicated state and that state is final (i.e., cannot be rolled back).

%We assume that the owner of \src is either running their own IPC Agent to perform the necessary operations described below, or uses another trusted IPC agent to act on their behalf.
Depositing \funcParam{amount} coins from an account \accountNameFull{P}{a} in the parent subnet \subnetName{P} to an account \accountNameFull{P/C}{b} in the child subnet \subnetName{P/C}
involves the following steps.

\begin{enumerate}

    \item The owner of \accountNameFull{P}{a} submits, using their wallet, a transaction\\
    \tx{tx} = \funcNameFull{P}{\saOf{C}}{Deposit}(\funcParam{amount}, \accountName{b}).
    
    \item \subnetName{P} orders and executes \tx{tx}, transferring \funcParam{amount} coins from \accountName{a} to \saOf{C}.
    
    \item \label{item:deposit-step-create-pof}
    When \subnetName{P}'s replicated state that includes \tx{tx} becomes final (for some subnet-specific definition of finality provable to \subnetName{P/C}.\gw, which contains the \pof verification logic),
    the \ipc agent constructs a {\pof}(\tx{tx}).
    
    \item \label{item:deposit-step-submit-pof}
    The \ipc agent submits a transaction\footnote{In a practical implementation, instead of submitting a separate transaction for each deposit,
    the \ipc agent may submit multiple deposits batched in a single transaction,
    with a single \pof proving the finality of the child state in which all the corresponding funds have been transfered to \saOf{C}.
    This optimizes both performance and cost (transaction fees).
    }\\
    \tx{tx'} = \funcNameFull{P/C}{\gw}{MintDeposited}(\funcParam{amount}, \accountName{b}, {\pof}(\tx{tx})).
    
    \item \subnetName{P/C} orders and executes \tx{tx'}, which results in minting \funcParam{amount} new coins and adding them to the balance of \accountNameFull{P/C}{b}.
    
\end{enumerate}

After all the above steps are performed, the newly minted \funcParam{amount} of coins at the child is backed by the analogous locked amount at \subnetName{P}.\saOf{C}.
However, those coins can effectively only be used by the owner of \accountNameFull{P/C}{b},
since \subnetName{P}.\saOf{C} will not transfer its coins within \subnetName{P} until they are burned in \subnetName{P/C} during a withdrawal operation (see below).

\begin{example}
\label{ex:deposit}

Imagine that a player wants to start using the gaming platform (running on the subnet \subnetName{P/C}), but only has funds in an account \accountNameFull{P}{a} in the parent subnet.
To be able to join games (L3) that require a collateral (as in \Cref{ex:create-game-subnet}), the player decides to fund their account \accountNameFull{P/C}{b} (L2) with 20 coins.
Thus, the player submits a \funcNameFull{P}{\saOf{C}}{Deposit(20, \accountName{b})} transaction, starts an \ipc agent process that performs steps \ref{item:deposit-step-create-pof} and \ref{item:deposit-step-submit-pof} above,
and waits until the funds appear on \accountNameFull{P/C}{b}.
\guy{Regarding colllateral, this example is a bit confusing to me. I thought collateral is to be held at the parent (at least for the suggested implementation).}\arp{I think that in this example, games that require collateral in this example are L3 not L2, but yes, we should maybe not use collateral here, but instead betting on games or something that is less confusing (chipping in a prize for the winner). I added L3 and L2 for clarity}
\end{example}

\subsection{Withdrawals}
\label{sec:withdraw}

A withdrawal is a transfer of funds from an account in the child subnet to an account in the parent subnet.
The following functions are exposed by the \ipc actors to enable withdrawals.
\begin{align*}
    \gw.&\funcName{Withdraw}(\funcParam{amount, account})\\
    \sa.&\funcName{ReleaseWithdrawn}(\funcParam{amount, account, \pof})
\end{align*}
The \funcParam{amount} is the amount of funds to be withdrawn, \funcParam{account} is the destination account in the parent subnet to which the withdrawn funds are to be credited, and \funcParam{\pof} is a proof of finality proving that \gw.\funcName{Withdraw}(\funcParam{amount, account}) has been applied to the child subnet's replicated state and that state is final (i.e., cannot be rolled back).

Withdrawing \funcParam{amount} coins from an account \accountNameFull{P/C}{b} in the child subnet \subnetName{P/C} to an account \accountNameFull{P}{a} in the parent subnet \subnetName{P}
involves the following steps.

\begin{enumerate}

    \item The owner of \accountNameFull{P/C}{b} submits, using their wallet, a transaction\\
    \tx{tx} = \funcNameFull{P/C}{\gw}{Withdraw}(\funcParam{amount}, \accountName{a}).
    
    \item \subnetName{P/C} orders and executes \tx{tx}, burning \funcParam{amount} coins from \accountName{b}.%
    
    \item When \subnetName{P/C}'s replicated state that includes \tx{tx} becomes final (for some subnet-specific definition of finality provable to \subnetName{P}.\saOf{C}),
    the \ipc agent constructs a {\pof}(\tx{tx}).
    
    \item The \ipc agent submits a transaction%
    \footnote{Like with deposits, withdrawals can also be batched for better performance and lower cost. Our implementation applies this optimization to both deposits and withdrawals, further combined with checkpoints (see~\cref{sec:ref-impl}).}\\
    \tx{tx'} = \funcNameFull{P}{\saOf{C}}{ReleaseWithdrawn}(\funcParam{amount}, \accountName{a}, {\pof}(\tx{tx})).
    
    \item \subnetName{P} orders and executes \tx{tx'}, which results in \subnetName{P}.\saOf{C} transferring \funcParam{amount} coins to account \accountNameFull{P}{a}.
    
\end{enumerate}

The above procedure ensures that the locked \funcParam{amount} at the parent is not released until the child has already burned the minted \funcParam{amount} of coins.
The \subnetName{P}.\saOf{C} actor ensures (by verifying the associated \pof) that the coins have been burned in \subnetName{P/C}
before releasing the corresponding \funcParam{amount} back into circulation in \subnetName{P}.

\begin{example}
\label{ex:withdrawal}
A player might want to stop using the gaming platform and withdraw all the funds back to the parent subnet, in order to spend them on something else.
They still have 20 coins on their account \accountNameFull{P/C}{b} that they want to transfer back to \accountNameFull{P}{a}.
The player performs the withdrawal by submitting a transaction \funcNameFull{P/C}{\gw}{Withdraw(20, \accountName{a})},
starts an \ipc agent process to perform the necessary inter-subnet communication, and waits until the coins arrive at \accountNameFull{P}{a}.
(The player might need to spend a part of the 20 coins on fees for both the \funcName{Withdraw} and the \funcName{ReleaseWithdrawn} transactions.)
\end{example}

\subsection{Checkpointing} 
Checkpointing is a method for a parent subnet to keep a record of the evolution of its child subnet's replicated state
by including snapshots of the child's replicated state (called checkpoints) in the parent's replicated state.
If, for some reason, the child subnet misbehaves as a whole (e.g., by a majority of its replicas being taken over by an adversary), 
agreement can be reached in the parent subnet about how to proceed.
For example, which checkpoint should be considered the last valid one,
and potentially used as the initial checkpoint (equivalent in concept to a "genesis block") for a new sanitized subnet.
The following function is exposed by the \sa to enable checkpointing.

\begin{align*}
    \sa.&\funcName{Checkpoint}(\funcParam{snapshot}, \pof)
\end{align*}

A checkpoint can be triggered by predefined events (e.g.,  periodically, after a number of state updates, triggered by a specific user or set of users, etc.).
The \ipc agent is configured with the (subnet-specific) checkpoint trigger, monitors the child subnet's replicated state,
and takes the appropriate action when the trigger condition is satisfied by the child subnet's state.
A checkpoint of subnet \subnetName{P/C} to its parent \subnetName{P} is created as follows:
\begin{enumerate}

    \item When the predefined checkpoint trigger is met in the replicated state of \subnetName{P/C},
    the \ipc agent retrieves the corresponding snapshot of \subnetName{P/C}'s replicated state (\emph{state}) from the child subnet
    and constructs the proof of its finality {\pof}(\emph{state}).

    \item The \ipc agent submits a transaction\\
    \tx{tx} = \funcNameFull{P}{\saOf{C}}{Checkpoint}(\funcParam{state}, {\pof}(\funcParam{state})).

    \item \subnetName{P} orders and executes \tx{tx}, which results in \subnetName{P}.\saOf{C} including \funcParam{state}
    (i.e., the checkpoint of \subnetName{P/C}'s replicated state) in its own actor state.

\end{enumerate}

\subsection{Propagating cross-net transactions}
\label{sec:cross-net-tx}

Cross-net transactions are a means of interaction between actors located on different subnets.
Unlike a "standard" transaction issued and submitted to a subnet by a user's wallet,
a cross-net transaction is issued by actors of another subnet.

Since those actors themselves are not processes (but mere parts of a subnet's replicated state),
they cannot directly submit transactions to other subnets.
IPC therefore provides a mechanism to propagate these transactions between subnets using the following functions of the \gw.

\begin{align*}
    \gw.&\funcName{Dispatch}(\funcParam{tx, src, dest})\\
    \gw.&\funcName{Propagate}(\funcParam{tx, src, dest, \pof})
\end{align*}

In a nutshell, if an actor's logic in subnet $\subnetName{S}_1$ produces a transaction for a different subnet $\subnetName{S}_2$,
it calls $\subnetName{S}_1$.\gw.\funcName{Dispatch}, which saves the transaction in $\subnetName{S}_1$'s \gw buffer that we call the \emph{\postoffice}.
IPC agents, monitoring the \postoffice, then iteratively submit the transaction to the appropriate next subnet along the path from $\subnetName{S}_1$ to $\subnetName{S}_2$ using \gw.\funcName{Propagate}.

Since, in general, we only rely on an IPC agent to be able to submit transactions to the parent or children of a subnet whose state it observes,
an IPC agent only propagates the transaction to the parent or child, depending on which is next along the shortest path from $\subnetName{S}_1$ to $\subnetName{S}_2$ in the IPC hierarchy.
After such ``one hop``, the transaction is again placed in the \postoffice of the parent / child, and the process repeats until the transaction reaches its destination subnet.

More concretely, we illustrate the propagation of a cross-net transaction using an example where an actor in subnet \subnetName{P/A}
is sending a cross-net transaction \tx{tx} to its "sibling" subnet \subnetName{P/B}.
\tx{tx} is first propagated from \subnetName{P/A} to its parent \subnetName{P}, which, in turn, propagates it to its other child \subnetName{P/B}.
We use the function \gw.\funcName{Dispatch} in a subnet to announce that the transaction is ready to be propagated
and the function \gw.\funcName{Propagate} to notify a subnet about a cross-net transaction to be passed on (or delivered, if the destination has been reached).

\begin{enumerate}

    \item An actor \subnetName{P/A}.\actorName{ActorA} constructs a transaction\\
    \tx{tx} = \funcNameFull{P/B}{ActorB}{SomeFunction}(\funcParam{someParams})

    \item \subnetName{P/A}.\actorName{ActorA} invokes the funcion \funcNameFull{P/A}{\gw}{Dispatch}(\tx{tx}, \subnetName{P/A}, \subnetName{P/B})
    (note that no additional transactions are necessary here).

    \item The implementation of \funcNameFull{P/A}{\gw}{Dispatch} adds \tx{tx} along with the routing metadata to a \subnetName{P/A}.\gw.\dataField{\postoffice}.

    \item \label{item:first-propagation} Let $state_\subnetName{A}$ be the state of subnet \subnetName{P/A} where \tx{tx} is already included in \subnetName{P/A}.\gw.\dataField{\postoffice}.
    When the \ipc agent responsible for the interaction between \subnetName{P/A} and \subnetName{P} detects that $state_\subnetName{A}$ is final,
    it constructs a {\pof}($state_\subnetName{A}$) and submits a transaction\\
    $\tx{tx}_\subnetName{A}$ = \funcNameFull{P}{\gw}{Propagate}(\tx{tx}, \subnetName{P/A}, \subnetName{P/B}, {\pof}($state_\subnetName{A}$)).

    \item Subnet \subnetName{P} orders and executes $\tx{tx}_\subnetName{A}$, verifying {\pof}($state_\subnetName{A}$)
    and (internally) invoking \funcNameFull{P}{\gw}{Dispatch}(\tx{tx}, \subnetName{P/A}, \subnetName{P/B}).
    This, in turn, adds \tx{tx} along with its routing metadata to \subnetName{P}.\gw.\dataField{\postoffice}.

    \item Analogously to step \ref{item:first-propagation}, the \ipc agent submits a transaction\\
    $\tx{tx}_\subnetName{P}$ = \funcNameFull{P/B}{\gw}{Propagate}(\tx{tx}, \subnetName{P/A}, \subnetName{P/B}, {\pof}($state_\subnetName{P}$)),
    where $state_\subnetName{P}$ is the state of \subnetName{P} with \tx{tx} already included in \subnetName{P}.\gw.\dataField{\postoffice}.

    \item Upon ordering and executing $\tx{tx}_\subnetName{P}$, \funcNameFull{P/B}{\gw}{Propagate} verifies {\pof}($state_\subnetName{P}$).
    Detecting that the destination is the own subnet, the implementation of \funcNameFull{P/B}{\gw}{Propagate} executes \tx{tx} instead of propagating it.

\end{enumerate}

\begin{example}

In our gaming example, imagine that a game (running in its own subnet that is a child of the gaming platform's subnet) has finished and the ranking of the involved players needs to be updated.
The game server is implemented as an actor on the game's own subnet, while the gaming platform (storing the player ranking tables) is an actor of its parent.
To update the ranking, the game actor would use a cross-net transaction to inform the platform actor about the results of the game and the platform actor would update the rankings accordingly.

\end{example}

\subsection{Removing a child subnet}
\label{sec:remove}


To remove child subnets, the \gw exposes the following function.
\begin{align*}
    \gw.&\funcName{ToRemove}()\\
    \gw.&\funcName{RemoveChild}(\funcParam{subnetName},\pof)
\end{align*}
The function \gw.\funcName{ToRemove}() marks the subnet to be removed in the \gw of the subnet, while \gw.\funcName{RemoveChild}(\funcParam{subnetName},\pof) effectively deregisters the subnet from the IPC hierarchy. Removing a child subnet \subnetName{P/C} from the IPC hierarchy is performed in the following steps:
\begin{enumerate}
\item A replica of \subnetName{P/C} submits a transaction \tx{tx}=\subnetName{P/C}.\gw.\funcName{ToRemove}(). The validity of this transaction to be included in the subnet's replicated state is subnet-specific, and may require coordination among the replicas to validate the removal of the subnet, as well as additional parameters to the function.
\item When \subnetName{P/C}'s replicated state that includes \tx{tx} becomes final, the \ipc agent constructs a {\pof}(\tx{tx}).
    
    \item The \ipc agent submits a transaction\\
    \tx{tx'} = \funcNameFull{P}{\gw}{RemoveChild}(\subnetName{P/C}, {\pof}(\tx{tx})).
    
    \item \subnetName{P} orders and executes \tx{tx'}, which results in \subnetName{P}.\gw deregistering \subnetName{P/C} from the IPC hierarchy.
\end{enumerate}

As the deposited funds are in the \sa, the subnet can still keep operating for all accounts to withdraw their funds back at the parent. 

% A child subnet $P/C$ can be removed from its parent $P$ through a transaction invoking $P.\gw.RemoveChild(P/C)$. \add{This transaction effectively deregisters the subnet from the IPC hierarchy. We detail further how to remove a child subnet for the reference implementation in~\cref{sec:ref-impl}.}

\subsection{Proof-of-stake subnets}
\label{sec:pos-subnets}

% \arp{I do not think we should particularize in PoS subnets. We just require a wrapper of any blockchain into single finality (with assumption on PoF for PoW-style blockchains after x amount of blocks). For example, for non-PoS children, the IPC agent can require joining and leaving as replica in order for a withdrawal/cross-net/checkpoint from a replica to be valid (even at the child itself, if replica in PoW child subnet creates block but replica not part of membership according to \sa, block is rendered invalid). Slashing can be performed on PoW replicas this way as well (i.e. time-lock withdrawals/release of funds $+$ synchrony). So these operations are not really PoS-specific. The only thing that is not immediately if not PoS is a direct mapping between voting rights (amount of work in PoW) and collateralized amount (this can be solvable limiting amount transferred in blocks for added security, or other measures, for example), but besides it being technically possible, it is also kind of ok since the collateral in PoW is anyways the cost to produce blocks.}
% \matej{I agree that, technically, some of the concepts around collateral and slashing could be, in some way, applied to non-PoS subnets,
% even though I'd argue that it'd be quite a stretch in practice in my opinion.
% Unless you strongly oppose calling this PoS features, I'd suggest staying with the current approach, mostly for time reasons,
% and we can refine / generalize later when we have other critical things sorted out.
% Especially since we do say that the staked collateral determines the voting power, I think it's completely fair to call it PoS.}\arp{Time shortage is a valid reason, but note that we do not require that collateral determines voting power (we do not even conceptually require a minimum amount of collateral at the parent for the child to run its consensus protocol, same way we do not require any specific slashing rule), which adds to the mnuch simpler requirement of single finality (or capped finalization with asumption of single finality, like in Filecoin as rootnet). Let us leave it like this for now and modify later if reused with more time} 

In order to disincentivize replicas of a subnet from misbehaving, \ipc provides a mechanism for conditioning a replica's participation in the child subnet on \emph{collateral} in a proof-of-stake fashion.
To this end, the \sa can associate each replica of the child subnet with a collateral.
Replicas must transfer this collateral to the \sa, and the \sa only releases the collateral back once the corresponding replica stops participating in the subnet. The way in which the collateral associated with replicas impacts the functioning of the child is subnet-specific.

If a child replica provably misbehaves, the proof of such misbehavior can be submitted as a transaction to the \sa (invoking its \emph{Slash} function).
The \sa then decreases the amount of collateral associated with the offending replica in accordance with its (subnet-specific) slashing policy.

Note that collateral is different from funds deposited for use in the child subnet.
Unlike the deposited funds, collateral is not made available in the child subnet
and stays in the parent's \sa until the associated replica stops participating in the subnet, either by leaving or by being slashed.

The advantage of this approach is that a child subnet can leverage funds in the parent subnet to serve as collateral.
In case of provable misbehavior even of replicas that have gained complete power over the child subnet (e.g., by having staked most of the collateral),
those replicas can still be slashed at the parent.
It also prevents long-range \cite{azouvi2022Pikachu} and similar attacks by maintaining membership information (sometimes also referred to as power table) of the child subnet at the parent.
The membership saved in the \sa's state defines the ground truth about what replicas the child subnet should consist of,
as well as their relative voting power (proportional to the staked collateral) in the underlying ordering protocol.
The child subnet observes the state of the \sa in the parent and, when the membership information changes, reconfigures accordingly.

Since we expect PoS-based child subnets to become very common,
\ipc provides the following native functionality for managing PoS-based subnets:
\begin{itemize}
    \item Manipulate the membership by staking and releasing collateral associated with replicas
    \item Slashing, where \ipc permanently removes / redistributes a part of the collateral associated with a provably misbehaving replica (e.g., one that sends conflicting messages in the ordering protocol)
\end{itemize}

\Cref{ex:create-game-subnet} provides a concrete case of where and how PoS-based subnets can be used.

\subsubsection{Staking collateral}
\label{sec:staking-collateral}

To increase a replica's collateral in a PoS-based subnet, \ipc uses the following functions.

\begin{align*}
    \sa.&\actorName{StakeCollateral}(\funcParam{account, replica, amount})\\
    \gw.&\actorName{UpdateMembership}(\funcParam{membership, \pof})
\end{align*}

Concretely, to increase the amount of collateral associated with a \funcParam{replica} in subnet \subnetName{P/C},
collateral must be staked for \funcParam{replica} in the parent \subnetName{P}'s \saFull \subnetName{P}.\saOf{C}.
Let the the account from which the collateral is transfered to \subnetName{P}.\saOf{C} be \accountNameFull{P}{a}.
(Any user with sufficient account balance, i.e. at least \funcParam{amount} and the transaction fee, can perform this operation.)

The child subnet, holding its own local copy of the target membership, is then informed (through an \ipc agent) about the membership change and reconfigures to reflect it.
Note that it is required for the child subnet to hold a copy of the membership in its replicated state, so that all its replicas observe it in a consistent way.
In fact, the child subnet replicated state contains two versions of membership information:
\begin{itemize}
    \item The \emph{target membership}, which is a local copy of the membership stored in \subnetName{P}.\saOf{C} and designates the desired membership of the child subnet
    to which the subnet must reconfigure.
    \item The \emph{current membership}, which is the actual membership currently being used by the subnet to order and execute transactions.
    Since the process of reconfiguration to a new membership is usually not immediate, the current membership may "lag behind" the target membership.
\end{itemize}
The whole staking procedure is as follows.

\begin{enumerate}

    \item The owner of \accountNameFull{P}{a} submits the transaction\\
    \tx{tx} = \funcNameFull{P}{\saOf{C}}{StakeCollateral}(\accountNameFull{P}{a}, \funcParam{replica}, \funcParam{amount}).

    \item After ordering \tx{tx}, \subnetName{P}.\saOf{C} increases the collateral associated with \funcParam{replica} by \funcParam{amount},
    which is deducted from the submitting user's account in \subnetName{P}.
    If no collateral has been previously associated with \funcParam{replica}, this effectively translates in \funcParam{replica} joining the subnet.

    \item \label{item:update-membership} The \ipc agent, upon detecting the updated \funcParam{membership} in \subnetName{P}.\saOf{C} through the application of \tx{tx},
    constructs a {\pof}(\tx{tx}) and submits the transaction\\
    \tx{tx'} = \funcNameFull{P/C}{\gw}{UpdateMembership}(\funcParam{membership}, {\pof}(\tx{tx})).

    \item Upon ordering \tx{tx'} and successfully verifying {\pof}(\tx{tx}), \subnetName{P/C}.\gw updates its target membership.

    \item Subnet \subnetName{P/C} reconfigures (the reconfiguration procedure is specific to the implementation of the subnet and its ordering protocol)
    to use the new \funcParam{membership} as its current membership.
    
\end{enumerate}

\subsubsection{Releasing collateral}

Releasing collateral works similarly (but inversely) to staking, with one significant difference.
Namely, once the child subnet \subnetName{P/C} reconfigures to reflect the updated membership, the parent's \saFull \subnetName{P}.\saOf{C} does not release the staked funds
until it is given a proof that the child subnet \subnetName{P/C} finished the (subnet-specific) reconfiguration procedure.
The following functions are involved in releasing collateral:

\begin{align*}
    \sa.&\actorName{RequestCollateral}(\funcParam{replica, amount, account})\\
    \gw.&\actorName{UpdateMembership}(\funcParam{membership, \pof})\\
    \sa.&\actorName{ReleaseCollateral}(\funcParam{membership, \pof})
\end{align*}

Let \accountNameFull{P}{a} be an account that has staked collateral for a \funcParam{replica} in a PoS-based subnet \subnetName{P/C}.
The procedure for releasing collateral is as follows.

\begin{enumerate}

    \item The owner of \accountNameFull{P}{a} submits the transaction\\
    \tx{tx} = \funcNameFull{P}{\saOf{C}}{RequestCollateral}(\funcParam{replica}, \funcParam{amount}, \accountName{a}),\\
    where \funcParam{amount} is the amount of funds the owner of \accountNameFull{P}{a} wants to reclaim.

    \item After ordering \tx{tx} and checking that \accountNameFull{P}{a} has indeed previously staked at least \funcParam{amount} for \funcParam{replica},
    \subnetName{P}.\saOf{C} decreases the collateral associated with \funcParam{replica} by \funcParam{amount}.
    Note that \subnetName{P}.\sa does not yet transfer \funcParam{amount} back to \accountNameFull{P}{a}

    \item The \ipc agent, upon detecting the updated \funcParam{membership} in \subnetName{P}.\saOf{C} through the application of \tx{tx},
    constructs a {\pof}(\tx{tx}) and submits the transaction\\
    \tx{tx'} = \funcNameFull{P/C}{\gw}{UpdateMembership}(\funcParam{membership}, {\pof}(\tx{tx})).

    \item Upon ordering \tx{tx'} and successfully verifying {\pof}(\tx{tx}), \subnetName{P/C}.\gw updates its target membership.

    \item Subnet \subnetName{P/C} reconfigures (the reconfiguration procedure is specific to the implementation of the subnet and its ordering protocol)
    to use the new \funcParam{membership} as its current membership.

    \item The \ipc agent detects that the current membership of \subnetName{P/C} has changed.
    Let \funcParam{state} be the the replicated state of \subnetName{P/C} where the current membership has already been updated.

    \item The \ipc agent constructs a {\pof}(\funcParam{state}) and submits the transaction\\
    \tx{tx''} = \funcNameFull{P}{\saOf{C}}{ReleaseCollateral}(\funcParam{membership}, {\pof}(\funcParam{state}))

    \item Upon ordering \tx{tx''} and successfully verifying {\pof}(\tx{tx}),
    \subnetName{P}.\saOf{C} verifies that the received \funcParam{membership} reflects the requested change in the collateral of \funcParam{replica}
    and, if this is the case, transfers \funcParam{amount} to \accountNameFull{P}{a}.
    
\end{enumerate}

\subsubsection{Slashing a misbehaving replica}
\label{sec:slash}
Slashing is a penalty imposed on provably malicious replicas in PoS-based subnets.
When a replica of a child subnet provably misbehaves, \ipc agents can report the misbehavior to its parent subnet,
which can take an appropriate (configured) action (e.g., confiscate a part of the replica's collateral).
The definition of what constitutes a provable misbehavior is subnet-specific.
An example of such misbehavior is sending equivocating messages in the subnet's ordering protocol, such as two conflicting proposals for the same block height.
\ipc exposes the following function to enable slashing:

\begin{align*}
    \sa.&\funcName{Slash}(\funcParam{replica, PoM})
\end{align*}

Slashing a misbehaving \funcParam{replica} of a PoS-based subnet \subnetName{P/C} proceeds as follows:

\begin{enumerate}
    
    \item The \funcParam{replica} provably misbehaves, e.g., by sending two signed contradictory messages
    that would not have been sent if \funcParam{replica} strictly followed its prescribed distributed protocol.

    \item An \ipc agent is informed of this misbehavior, e.g., by the replicas that received the contradictory messages,
    constructs a Proof of Misbehavior (\funcParam{PoM}) (e.g., a data structure containing the two contradictory messages signed by \funcParam{replica}),
    and submits the transaction \tx{tx} = \funcNameFull{P}{\saOf{C}}{Slash}(\funcParam{replica}, \funcParam{PoM})

    \item Upon ordering \tx{tx}, \subnetName{P}.\saOf{C} evaluates the \funcParam{PoM} against \funcParam{replica}
    and adapts its associated collateral accordingly, resulting in a new membership for \subnetName{P/C}.

    \item Updating the membership of \subnetName{P/C} and subsequent reconfiguration proceeds exactly as in \Cref{sec:staking-collateral}, from \Cref{item:update-membership} on.

\end{enumerate}

\begin{example}
Imagine the setting from \Cref{ex:create-game-subnet}, where a player creates a PoS-based subnet for a game of 4 players with a minimal collateral of 10 coins.
Suppose that one of the players makes a move in the game, but later realizes that it would be beneficial to revert that move.
What is more, the majority of other players would also benefit from the game state being reverted to the state just before the move occurred.
Those players might collude and agree off-band to revert their replicas of the game server to an older state,
and propose (in the child's ordering protocol) a different transaction to be ordered instead of the one containing the original game move.

However, the system (concretely, the \saFull) is configured such that the initial proposal of the original transaction,
together with the new proposal of the ``replacement'' transaction (both signed by the proposing replica),
constitute a \pom that can be verified by the \sa.
Any honest player that locally logs all proposals can thus submit a \funcNameFull{P}{\sa}{Slash} transaction, with the offending replica and the \pom as arguments,
receiving (in the parent subnet) a part of the collateral associated with the offending replica.
Moreover, if the child subnet's protocol allows to prove that a replica supported
(in some protocol-specific way -- imagine signed PBFT prepare messages)
more than one proposal for the same block height, the other colluding replicas can also be slashed.
\end{example}
