\section{Example Use Case: Chess Platform}
\label{sec:example-use-case}

To better understand how \ipc works and how it is useful, let us expand on the example application of a distributed chess platform sketched in the introduction of this document.
Imagine platform where registered chess players meet and play against each other, while the platform maintains player rankings (e.g. Elo ratings).
Tournaments can be organized as well, where each participant pays a participation fee and the winner(s) obtain prize money (both in form of coins).
We now describe how \ipc could be used to build this hypothetical application in a fully distributed fashion.

\paragraph{Rootnet with all users' funds (\subnetName{L1}).}
The rootnet is used as a financial settlement layer.
Most of users' coins are on accounts residing in the rootnet's replicated state.
A robust established blockchain system like Filecoin would be a good candidate for use as the rootnet.
Its relatively higher latency and lower throughput (that is often the price for security and robustness) is not a practical issue,
as users will rarely directly interact with it.

\paragraph{Chess platform as a subnet (\subnetName{L2}).}
The functionality of the chess platform
(such as maintaining score boards, recommending opponents to players, or organizing tournaments)
is implemented as a distributed application on a dedicated subnet.
This subnet uses a significantly faster BFT-style consensus protocol (such as Trantor),
since the application needs to be responsive for the sake of user experience,
and deals, in general, with significantly fewer funds than the rootnet (only as much as users dedicate to playing chess).
The replicas constituting this subnet are run by chess clubs or even some (not necessarily all) individual chess players
(who do not necessarily trust each other, e.g. to not manipulate the score boards).
To have a replica in the \subnetName{L2} subnet, the club (or the player) need to lock a certain amount of funds as collateral that can be slashed by the system in case of the replica misbehaving.
The collateral / slashing mechanism is described in more detail in \Cref{sec:incentives-collateral-slashing,sec:slash}.

\paragraph{Individual games (\subnetName{L3}).}
For each individual game of chess, a new child of the \subnetName{L2} subnet is created (\Cref{sec:create}).
Since not much is usually at stake in a single game and only two players are involved,
the whole \subnetName{L3} subnet may even be implemented by a single server that both players trust.
However, this decision is completely up to the players and they may choose a different implementation of the \subnetName{L3} subnet
when starting the game (by submitting the corresponding transactions to the \subnetName{L2} subnet).
A chess game is also very easily modeled as a simple application, its state consisting of the positions of the individual pieces on the board,
while players' moves are represented as transactions.
When the game finishes, its result is automatically reported to the \subnetName{L2} subnet (\Cref{sec:cross-net-tx}), which updates the players' ratings accordingly,
and the \subnetName{L3} subnet is disposed of (\Cref{sec:remove}).

\paragraph{Player accounts.}
Each player has an account on the \subnetName{L2} subnet where they deposit funds (\Cref{sec:deposit}) from the rootnet by submitting a corresponding \subnetName{L1} transaction%
\footnote{We call a transaction submitted to the \subnetName{L1} blockchain an ``\subnetName{L1} transaction''.}.
They use these funds to pay transaction fees on the \subnetName{L2} subnet and tournament registration fees.
A player can transfer funds back to their \subnetName{L1} account through a withdraw operation (\Cref{sec:withdraw}) -- again, by submitting an \subnetName{L2} transaction.

\paragraph{Tournaments.}
Chess tournaments can be organized using the platform, where each player registers by submitting a corresponding \subnetName{L2} transaction.
When the tournament finishes, the winner receives the prize money (obtained through the registration fees) on their \subnetName{L2} account.
One can also easily imagine that only part of the collected fees transforms to the prize, while the rest can remain in the platform
and be used for other purposes, such as rewarding the owners of the replicas running the subnet that hosts the platform (i.e., the \subnetName{L2} subnet).

\paragraph{}
This simple use case utilizes most of \ipc's features.
Throughout the rest of the document, we will use the on-line chess platform as a running example when describing \ipc's functionality in more detail.
\matej{This ``running example'' part is still to be added to the rest of the document. Coming soon, but I'd prefer to have some feedback on the general suitability of this example. Then I start integrating it in the text throughout the document.}