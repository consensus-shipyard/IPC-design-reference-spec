 \section{\ipc Actor Implementation}
 \label{sec:components}

\matej{This section has been moved around and needs updating. Read with care or don't read yet. Update coming soon.}

This section describes, at a high level, the implementation of \ipc's main components: the \gw, \sa, and the \ipc agent.

\subsection{\gwFull (\gw)}
\label{sec:gw}

The \gw is an actor that exists in every subnet in the \ipc hierarchy and contains all information and logic the subnet itself needs to hold in order to be part of \ipc.
The functionality of the \gw described in \Cref{sec:functionality} is summarized in \Cref{alg:gw}.
The \gw holds:
\begin{itemize}
    \item The names of its own, its parent's and its children's subnets

    \item The predicate used to evaluate the validity of \pofsFull.
    This predicate will be applied to {\pof}s from the parent subnet.
    It is specific to the subnets (and the protocols they use) involved in interactions with this subnet.

    \item The \postoffice storing all the outgoing cross-net transactions, along with their routing metadata (original source and ultimate destination subnets).
    We model the \postoffice as an infinitely growing set, from which the appropriate \ipc agents select only those elements that need to be submitted to other subnets.
    A garbage-collection mechanism for deleting delivered outgoing cross-net transactions from the sender subnet's state is out of the scope of this document.
    One can imagine, however, a garbage-collection mechanism based on acknowledgments (that are themselves cross-net messages).

    \item In a PoS-based subnet whose membership is managed by its parent, the \gw also contains the target membership that the subnet must reconfigure to (if the subnet is not using it yet).
    This membership is the subnet's local copy of the membership stored in its corresponding \saFull in the parent.
    It must be part of the subnet's replicated state, so that its replicas have a consistent view of it and can correctly reconfigure.
    Since reconfiguration does not happen immediately, the actual membership (also part of the subnet's replicated state) lags behind the target membership.

\end{itemize}

\begin{algorithm}[ht]
\footnotesize
\caption{\gwFull (\gw)}\label{alg:gw}
  \DontPrintSemicolon
  \SetKwProg{Component}{$\blacktriangleright$ \bf}{:}{\KwRet}
  \SetKwFor{UponKW}{}{}{fintq}
  \SetKw{Trigger}{trigger}
    \var{ownSubnetName}: name of the subnet the \gw resides in\;
    \var{parentSubnetName}: name of the parent subnet\;
    \var{childSubnets}: set of subnet names, initially empty\;
    \var{valid}: predicate over a \pof defining its validity criteria\;
    \var{\postoffice}: set of tuples (\var{transaction}, \var{source}, \var{destination}), initially empty\;
    \var{targetMembership}: the membership this subnet should reconfigure to if it is not yet using it (PoS only)\;
    \;
  
    \UponKW{\funcName{CreateChild}(\funcParam{name, params})}{
        newSubnetActor(\funcParam{params})\;
        \var{childSubnets} = \var{childSubnets} $\cup$ \{\funcParam{name}\}
    % Creates a new \sa with the given name and subnet-specific parameters (such as initial membership, etc.).
    % The subnet governed by the created \sa will be considered the child of the subnet of this \gw.
        % \replace{Creates a new \sa with the given name and subnet-specific parameters (such as initial membership, etc.).
        % The subnet governed by the created \sa will be considered the child of the subnet of this \gw.}{Registers subnet on IPC}
    }
    \UponKW{\funcName{RemoveChild}(\funcParam{name})}{
        \var{childSubnets} = \var{childSubnets} $\setminus$ \{\funcParam{name}\}
        % \matej{Is it meaningful to have this functionality at all?}%\arp{TLDR: killing a subnet is not necessarily trivial.\\ In our current implementation a client of a child subnet assumes $f<n/3$ where $n$ is the validators set at the child, but this is not necessarily a requirement of subnets. For example, state (lightning, payment) channels should be a specific case of subnets in which the set of validators = all clients of the subnet. In state channels, validators are required to sign in order to change the state (full safety but no liveness if $f\geq 1$) -- except for killing the subnet, allowing correct validators to withdraw with latest checkpoint without deadlocking their state in the corrupted subnet. This can be done with timelocks in an orderly manner. AFAIK Plasma chains even have the same functionality for clients to be able to circumvent a corrupted child subnet they belong to and skip to parent. }
    }
    \UponKW{\funcName{MintDeposited}(\funcParam{amount, account, \pof})}{
        \If{\normalfont{\var{valid}}(\pof)}{
            mint \funcParam{amount} new coins\;
            transfer minted coins to \funcParam{account}
        }
        % If \emph{\pof} is valid, adds \emph{amt} newly minted coins to account \dest.
        % A valid \pof means that a corresponding \emph{SA.Deposit(..., amt, dest)} has been successfully invoked in the parent's replicated state.
    }
    \UponKW{\funcName{Withdraw}(\funcParam{amount, account})}{
        \If{\funcParam{account}.balance $\geq$ \funcParam{amount}}{
            Burn \funcParam{amount} coins from \funcParam{account}
        }
        % Burns \emph{amount} coins from account \emph{src} to be returned to the \dest account in the parent subnet.
    }
    \UponKW{\funcName{Dispatch}(\funcParam{tx, src, dest})}{
        \var{\postoffice} = \var{\postoffice} $\cup$ \{(\funcParam{tx, src, dest})\}
        % Adds the cross-net transaction \emph{tx} to the list of transactions to be submitted to another subnet.
        % The IPC agents observing the state of this subnet will pick it up from here and perform the actual submission.
    }
    \UponKW{\funcName{Propagate}(\funcParam{tx, src, dest, \pof})}{
        \If{\normalfont{\var{valid}}(\pof)}{
            \If{\funcParam{dest} = \var{ownSubnetName}}{
                execute \funcParam{tx}
            }
            \ElseIf{$\exists s \in \var{childSubnets} \cup \{parentSubnetName\}: s$ \normalfont{is part of \funcParam{dest}}}{
                \funcName{Propagate}(\funcParam{tx, src, dest})
            }
            
        }
        % Processes an incoming cross-net transaction (subnitted by an \ipc agent) and propagates it to the next subnet.
        % If \tx{tx} is destined for the current subnet, executes it.
    }
    \UponKW{\funcName{UpdateMembership}(\funcParam{membership, \pof})}{
        \If{\normalfont{\var{valid}}(\pof)}{
            \var{targetMembership} = \funcParam{membership}
        }
        % Adds the cross-net transaction \emph{tx} to the list of transactions to be submitted to another subnet.
        % The IPC agents observing the state of this subnet will pick it up from here and perform the actual submission.
    }
 %    \add{\UponKW{Slashed($\mathcal{M}$, \pom, \pof)}{ 
 %         If \pom is a valid proof of misbehavior of a set $\mathcal{M}$ of validators, and if \pof is valid, then update state to reflect predefined punishment to misbehaviors in $\mathcal{M}$.
 % }}
\end{algorithm}

\subsection{\saFull (\sa)}
\label{sec:sa}

The \saFull (\sa) is the actor in the parent subnet's replicated state that governs a single child subnet.
It stores all information about the child subnet that the parent needs and logic that manipulates it.
The \sa is created by the \gw by invoking the parent's \gw.\funcName{CreateChild}(\funcParam{subnetName, params}) function (see \Cref{sec:create}).
The \funcParam{params} value plays a crucial part in the creation of the \sa, as it defines several parts of of the \sa's state.
The functionality of the \sa described in \Cref{sec:functionality} is summarized in \Cref{alg:sa}.
The \sa holds:
\begin{itemize}
    
    \item The predicate (\var{valid}) used to evaluate the validity of \pofsFull of the child subnet's replicated state.
    It is specific to the child subnet and the protocol it uses, and its definition is part of \funcParam{params} passed to \gw.\funcName{CreateChild} when the \sa is created.

    \item The amount of funds that are locked for use in the child subnet (\var{lockedFunds}).
    Deposits increase and withdrawals decrease this value accordingly.
    Keeping track of this value is only necessary for enforcing the firewall property, since a misbehaving child subnet might claim to withdraw more than has been deposited in it.
    Thus, before withdrawing, the \sa consults this value to make sure that the total amount of withdrawals never exceeds the amount previously deposited.

    \item Snapshots of the child subnet's replicated state obtained through invocations of the \funcName{Checkpoint} function (\var{checkpoints}).

    \item If the child subnet is a PoS-based one, the \sa also contains state required for managing the subnet's membership and the associated collaterals.
    The high-level implementation presented in \Cref{alg:sa} presents a simplified view of this state and the associated logic, as it conveys the mechanisms involved without getting lost in details.
    In particular, the presented description neglects some corner cases arising from concurrent handling of multiple staking, releasing, and/or slashing procedures.
    In a real-world implementation, however, these corner cases can easily be addressed.

    The \sa stores information on which child replica is has how much collateral (\var{childMembership}),
    how much collateral (and for which replica) is staked from which account (\var{collateral}),
    and which accounts requested the withdrawal of how much collateral (\var{collateralRequests}).
    Moreover, the \sa's state contains a predicate for checking the validity of proofs of misbehavior (\var{validPom})
    and a procedure to execute when a valid PoM is received through the \funcName{Slash} function.
    Both \var{validPoM} and \var{slashingPolicy} are part of \funcParam{params} passed to \gw.\funcName{CreateChild} when the \sa is created.
    
\end{itemize}


\begin{algorithm}[ht]
\footnotesize
\caption{\saFull (\sa)}\label{alg:sa}
  \DontPrintSemicolon
  \SetKwProg{Component}{$\blacktriangleright$ \bf}{:}{\KwRet}
  \SetKwFor{UponKW}{}{}{fintq}
  \SetKw{Trigger}{trigger}
    \var{valid}: predicate over a \pof defining its validity criteria\;
    \var{lockedFunds}: total amount of funds circulating in the child subnet\;
    \var{checkpoints}: set of checkpoints of the child's replicated state\;
    \var{childMembership}: map of replica identities to their respective staked collaterals\;
    \var{collateral}: map of accounts to replica identities, to staked collaterals\;
    \var{collateralRequests}: set of received but unsatisfied requests for releasing collateral\;
    \var{validPoM}: predicate over a PoM defining its validity criteria\;
    \var{slashingPolicy}: procedure to execute on reception of a valid PoM\;
    \;
  
    \UponKW{\funcName{Deposit}(\funcParam{amount, account})}{
        \var{lockedFunds} += amount\;
    }
    \UponKW{\funcName{ReleaseWithdrawn}(\funcParam{amount, account, PoF})}{
        \If{\normalfont{\var{valid}}(\pof) $\land$ \var{lockedFunds} $\geq$ \var{amount}}{
            \var{lockedFunds} $-=$ \funcParam{amount}\;
            transfer \funcParam{amount} to \funcParam{account}\;
        }
    }
    \UponKW{\funcName{Checkpoint}(\funcParam{snapshot, \pof})}{
        \If{\normalfont{\var{valid}}(\pof)}{
            \var{checkpoints} = \var{checkpoints} $\cup$ \{\funcParam{snapshot}\}\;
        }
    }
    \UponKW{\funcName{StakeCollateral}(\funcParam{account, replica, amount})}{
        \var{childMembership}[\funcParam{replica}] += \funcParam{amount}\;
        \var{collateral}[\funcParam{account}][\funcParam{replica}] += \funcParam{amount}\;
    }
    \UponKW{\funcName{RequestCollateral}(\funcParam{replica, amount, account})}{
        \If{\var{collateral}\normalfont{[}\funcParam{account}\normalfont{][}\funcParam{replica}\normalfont{]} $\geq$ \funcParam{amount}}{
            \var{childMembership}[\funcParam{replica}] $-=$ \funcParam{amount}\;
            \var{collateralRequests} = \var{collateralRequests} $\cup$ \{(\funcParam{amount, account})\}\;
        }
    }
    \UponKW{\funcName{ReleaseCollateral}(\funcParam{membership, \pof})}{
        \If{\normalfont{\var{valid}}(\pof) $\land$ \funcParam{membership} = \var{subnetMembership}}{
            \For{(\funcParam{amount, account}) $\in$ \var{collateralRequests}}{
                transfer \funcParam{amount} to \funcParam{account}\;
            }
        }
    }
    \UponKW{\funcName{Slash}(\funcParam{replica, PoM})}{
        \If{\normalfont{\var{valid}}(\funcParam{PoM})}{
            \var{slashingPolicy}(\funcParam{PoM})
        }
    }
\end{algorithm}

    % OLD VERSION
    % The state of the \sa includes:
    % \begin{itemize}

    %     \item \emph{Accounting data.}
    %     This data describes the value that has been deposited to the child.
    %     It is considered locked inside the SA until it is withdrawn from the child.
    %     This data might consist of just a single value representing the sum of all such coins (``aggregated accounts'' approach), \jorge{maybe just use omnibus account as in traditional banking?}
    %     but might also contain finer-grained information about balances for each account in the child subnet (``segregated accounts'' approach).
    %     %We continue with the non-custodial approach as the other can be viewed as a specific limitation of it.

    %     \item \emph{Subnet consensus data.}
    %     This is the data (or a pointer thereto) that is needed to run the ordering protocol of the child subnet.
    %     It is specific to the ordering protocol but generally expected to contain information such as the ordering protocol used by the subnet, subnet configuration data such as the validator set, voting rights, and collateral deposits, and subnet governance mechanisms such as transaction fees, block rewards, and conditions for participation.         \TODO{Mention our reference implementation as a concrete example here, saying what information it stores.}

    %     \item \emph{Child state finality verification.} Logic to verify that a given child's replicated state is final,
    %     or that a particular \tx has been definitively included in the child's state.
    %     We expect that this logic will verify a \pof submitted (through transactions) by one or more IPC agents to the \sa.
    %     % For this, we will use the function \sa.\verifyGfinal{\tx}{\prf} which excepts as arguments a transaction (or state) and a \prf, and outputs True if \tx is considered globally final in the child subnet and False otherwise. This function must only depend on its inputs and the internal state of~\sa. For example, \prf is a threshold signature that can be verified against the set of validators in~\sa.
          
    %     \item \emph{Slashing data.} List of slashable misbehaviors and corresponding definition of what constitutes a valid proof of misbehavior (\pom), as well as penalties for misbehavior and rewards for reporting \pom and logic performing the actions necessary for slashing in the parent subnet.
    %     \item \emph{Checkpointing.} Child subnet's checkpoint data, or a pointer thereto, and checkpoint validity rules (and logic enforcing them).
    % \end{itemize}

% \subsection{IPC Agent}

% We assume that an IPC Agent is only responsible for a single pair of parent-child subnets, the state of which it can access.
% It reacts to changes in those states triggered by the corresponding invocations of \actors, as listed below.

% \begin{algorithm}[H]
% \footnotesize
% \caption{IPC Agent interface}\label{alg:agent-interface}
%   \DontPrintSemicolon
%   \SetKwProg{Component}{$\blacktriangleright$ \bf}{:}{\KwRet}
%   \SetKwFor{UponKW}{upon}{do}{fintq}
%   \SetKw{Trigger}{trigger}
%   \Component{IPC Agent}{
%     \UponKW{parent.SA.Join(identity, src, collateral)}{
%         Constructs a \pof proving that the invocation of \emph{parent.SA.join(identity, src, collateral)} has been finalized in the parent's replicated state,
%         as well as subnet-specific metadata based on the \emph{identity} of the joining replica and the associated \emph{collateral}
%         and notifies the child subnet by submitting a transaction that invokes \emph{child.\gw.Joined(identity, metadata, \pof)}.
%     }
%      \UponKW{child.\gw.Leave(identity)}{
%         Constructs a \pof proving that the invocation of \emph{child.\gw.Leave(identity)} has been finalized in the child's replicated state
%         and notifies the parent subnet by submitting a transaction that invokes \emph{parent.SA.Left(identity, \pof)}
%      }
%     \UponKW{parent.SA.Deposit(..., amt, dest)}{
%         Constructs a \pof proving that the invocation of \emph{parent.SA.Deposit(src, amt, dest)} has been finalized in the parent's replicated state
%         and notifies the child subnet by submitting a transaction that invokes \emph{child.\gw.Deposited(amt, dest, \pof)}
%     }
%     \UponKW{child.\gw.Withdraw(..., amt, dest)}{
%         Constructs a \pof proving that the invocation of \emph{child.\gw.Withdraw(..., amt, dest)} has been finalized in the child's replicated state
%         and notifies the parent subnet by submitting a transaction that invokes \emph{parent.SA.Withdrawn(amt, dest, \pof)}
%     }
%     \UponKW{child.\gw.cross-netTX(tx, destName)}{
%         If \emph{destName} points up the hierarchy, submits \emph{tx} to the parent subnet.
%     }
%     \UponKW{parent.\gw.cross-netTX(tx, destName)}{
%         If \emph{destName} points down the hierarchy, submits \emph{tx} to the child subnet.
%     }
%     \add{
%     \UponKW{checkpoint condition in child}{
%         Create checkpoint, a \pof of the checkpoint at the child and submit from child to parent
%     }
%     \UponKW{misbehavior from set $\mathcal{M}$ found}{
%         Create a proof of misbehavior \pom and submit to parent
%     }
%     \UponKW{parent.\sa.Slash($\mathcal{M}, \pom$)}{
%         Constructs a \pof proving that the invocation of \emph{parent.\sa.Slash($\mathcal{M}, \pom$)} has been finalized in the parent's replicated state
%         and notifies the child subnet by submitting a transaction that invokes \emph{child.\gw.Slashed($\mathcal{M}, \pom, \pof$)}
%     }
%     }
% }
% \end{algorithm}

% OLD VERSION
% Note the function pairs \emph{Joined/Leave} of the \gw actor (\Cref{alg:gw-interface}) and \emph{Join/Left} of the \sa (\Cref{alg:sa-interface}).
% This is because the intended invocation pattern for several functionalities is as follows (to be detailed in \Cref{sec:functionality}).
% \begin{enumerate}
%     \item The initiating subnet invokes a function on a \actor (e.g., \emph{\sa.Join}).
%     \item IPC agent notices the invocation, constructs the required \pof, and submits a transaction to the other subnet (e.g., \emph{\gw.Joined})
% \end{enumerate}
