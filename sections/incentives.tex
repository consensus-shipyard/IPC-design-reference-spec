\section{Incentives}
\label{sec:incentives}

In the previous sections, we defined the components of an IPC system and their roles in implementing the IPC functionality.
Most of the functionality involved submitting transactions to subnets by an IPC Agent.
However, in general, submitting transactions (and their subsequent execution by the subnet) is associated with a \emph{cost} (often referred to as "gas").
We refer to the cost associated with a transaction as the \emph{transaction fee}, measured in coins.
An IPC Agent might need an incentive to participate in such a costly protocol.

Moreover, the replicas of a subnet might need to cooperate with IPC agents during the construction of \pofsFull.
Unless non-cooperation can be detected and penalized (see \Cref{sec:slashing}), subnet replicas also might need an incentive to participate in the creation of a \pof.

This section describes mechanisms that can be used to incentivize participants running IPC agents
to submit the required transactions and pay the corresponding transaction fees,
as well as replicas to participate in \pof creation.
It is \emph{not} the goal of this section to provide a game-theoretic model of viable incentive mechanisms and their analyses.
We merely present tools for implementing such mechanisms, to be used by those who design and implement concrete instances of IPC subnets.


\subsection{Accounts}

We assume that each IPC agent has accounts in both the parent and the child subnet
and that the fees for the transaction the IPC agent submits to the respective subnets are deducted from the respective accounts.
If the balance of the account is insufficient to pay the transaction fee, the transaction is considered invalid and is ignored by the subnet.
We further assume that smart contracts (the \sa and the \gw) can also hold funds and that their logic can distribute those funds among other accounts on their respective subnets.

\paragraph{Gateway Actor.}
The Gateway actor accumulates from its own subnet. For example, the subnet's implementation can require a certain part of each transaction fee to be sent to the Gateway Actor.

\paragraph{Subnet Actor.}
The Subnet Actor accummulates funds from the subnet it governs.
There are several ways how one can imagine the \sa to be funded,
for example, by withdrawals initiated by the child subnet's \gw,
or by periodicaly charging the child's replicas for being included in the replica set.

\subsection{Refunds and Rewards}

An IPC Agent submitting a transaction that invokes a function of the \gw or the \sa can receive a refund of the transaction fee directly from the invoked smart contract,
according to arbitrary rules defined in the smart contract's logic.
For example, such a rule could be to credit the submitting IPC Actor's account with the amount of the transaction fee, augmented by a fixed (or proportional) reward.

To incentivize the replicas of a subnet to collaborate with the IPC agent on the creation of \pofsFull, a similar mechanism can be deployed.
For example, a valid \pof would include metadata, where the replicas that participated in its creation could insert an address to receive a reward when the \pof is accepted.